\chapter{The Lifecycle and Cloud Radiative Effect of Deep Convective Clouds Over Africa} \label{chp:radiative_effect}

\section{Introduction}  %% \introduction[modified heading if necessary]
\acrshort{dcc} play a key role in the tropical atmosphere over Africa. 
Forming the ascending branch of the Hadley cells near the equator, \acrshort{dcc}s are critical to the circulation and heat transfer of the tropics \citep{riehl_heat_1958, weisman_mesoscale_2015}. 
The \acrshort{itcz} and its location is vital in determining the seasonal cycles of rainfall over central and western Africa \citep{nicholson_revised_2009, nicholson_itcz_2018}.
Understanding the behaviour of \acrshort{dcc}s over Africa has the potential for major impacts on the atmosphere, weather and society.

\acrshort{dcc}s also exert a key influence on the temperature of the tropics through their \acrshort{cre}. 
Due to their size, height and depth, \acrshort{dcc} anvils have large radiative effects in both the \acrshort{sw} and \acrshort{lw}, with both having average magnitudes that exceed 100\,\unit{W m^{-2}} \citep{hartmann_tropical_2016, wall_balanced_2018}. 
Due to the opposite signs of these two components, the average anvil \acrshort{cre} in the tropics is approximately zero \citep{ramanathan_cloud-radiative_1989, hartmann_effect_1992, stephens_cloudsat_2018}. 
Much of the focus on the anvil \acrshort{cre} feedback to global warming has been placed on the \acrshort{lw} response (see section~\ref{sec:anvil_feedbacks}, and in particular only consider the net response of anvils to large scale the tropical atmosphere, rather than the mechanisms affecting individual \acrshort{dcc}s.
In particular, the sensitivity of anvil \acrshort{cre} to changes in the diurnal cycle of convection has recieved little attention.
Previous research has highlighted that changes in the diurnal cycle of convection over Africa may lead to changes in \acrshort{cre} of \textpm 10\,\unit{Wm\textsuperscript{-2}} \citep{nowicki_observations_2004}.
Further investigation into the response of anvil \acrshort{cre} to changes in the diurnal cycle highlighted the need for cloud tracking approaches to study \acrshort{cre} over the anvil lifetime \citep{bouniol_macrophysical_2016, bouniol_life_2021}.


In this chapter, the novel cloud tracking methodology presented in chapter~\ref{chp:tracking_method} is used in conjunction with derived all-sky and clear-sky radiative fluxes to characterise the \acrshort{cre} over the lifecycles of individual anvil clouds. 
This methodology is applied to 4 months of data produced for the \acrshort{esa} Cloud-\acrfull{cci}+ project over sub-Saharan Africa. 
This dataset allows us to investigate both the \acrshort{cre} of individual \acrshort{cre}s, as well as the net anvil \acrshort{cre} over the entire region. 
The overall distribution of anvil \acrshort{cre} is found to be determined by the relationship between \acrshort{dcc} lifecycle and the diurnal cycle of the \acrshort{sw} \acrshort{cre}.
This has important implications for the response of \acrshort{dcc}s to a changing climate, as previously the impact of changes in the diurnal cycle of anvil \acrshort{cre} has seen little study.


\begin{table}[b]
\begin{tabular}{lllcc}
\tophline
Channel & Wavelength (\unit{\mu m}) & Description & Tracking & Retrieval\tabularnewline
\middlehline
1 & 0.64 & Visible & & \checkmark\tabularnewline
2 & 0.81 & \acrshort{nir} & & \checkmark\tabularnewline
3 & 1.64 & \acrshort{nir} & & \checkmark\tabularnewline
4 & 3.92 & \acrshort{nir} Window & & \checkmark\tabularnewline
5 & 6.25 & Upper troposphere \acrshort{wv} & \checkmark & \checkmark\tabularnewline
6 & 7.35 & Lower troposphere \acrshort{wv} & \checkmark & \checkmark\tabularnewline
7 & 8.70 & Mid-IR window & &\tabularnewline
8 & 9.66 & Ozone & &\tabularnewline
9 & 10.8 & Clean \acrshort{lw} window & \checkmark & \checkmark\tabularnewline
10 & 12.0 & Dirty \acrshort{lw} window & \checkmark & \checkmark\tabularnewline
11 & 13.4 & CO\textsubscript{2} & & \checkmark\tabularnewline
12 & 0.6--0.9 & High-resolution visible & &\tabularnewline
\bottomhline
\end{tabular}
% \belowtable{}
\caption{\acrshort{seviri} channels and their use in the \acrshort{dcc} tracking algorithm and cloud properties retrieval.
}
\label{table:seviri_channels}
\end{table}


\section{Data}


For this case study, data is used from \acrshort{seviri} \citep{aminou_msg_2002} aboard the \acrfull{msg} Meteosat-11 satellite, which is in a geostationary orbit above the equator at 0\textdegree W. 
Data from 4 months (May--August 2016) over sub-Saharan Africa (approximately 18\,\textdegree W--46\,\textdegree E, 31\,\textdegree S--15\,\textdegree N) is used at the full resolution of \acrshort{seviri} (3\,\unit{km} at nadir), along with retrieved cloud properties and derived broadband fluxes produced by the \acrshort{esa} Cloud-\acrshort{cci}+ project.
\acrshort{bt} from \acrshort{seviri} is used by the tracking algorithm, and reflectances and \acrshort{bt} are used by the cloud retrieval.

\acrshort{seviri} is a visible and \acrshort{ir} radiometer with a nadir spatial resolution of 3\,\unit{km} and a temporal sampling time of 15 minutes for the full earth disc. 
\acrshort{seviri} has 12 channels across the visible, \acrshort{nir} and thermal-IR spectrum, with one being a high-resolution visible channel with a nadir resolution of 1\,\unit{km}. 
A brief overview of these channels, along with which are used for tracking \acrshort{dcc}s and the cloud properties retrieval, is provided in table~\ref{table:seviri_channels}.


\begin{figure}[tp]
    \includegraphics[width=\textwidth]{figures/chapter4_01.png}
    \caption[
    Example observations from the Meteosat \acrshort{seviri} instrument at 15:00:00 \acrshort{utc} on 2016/6/01
    ]{
    Example observations from the Meteosat \acrshort{seviri} instrument at 15:00:00 \acrshort{utc} on 2016/6/01. a: A visible composite formed using the 1.6, 0.81 and 0.64\,\unit{\mu m} channels as the \acrshort{rgb} channels respectively, with 10.8\,\unit{\mu m} \acrshort{bt} during the night-time. The scene shows a cluster of cold cloud tops (cyan) over central Africa and the Southern Atlantic. b: 10.8\,\unit{\mu m} \acrshort{bt}. c: \acrshort{wvd} formed by the 6.3\,\unit{\mu m} channel minus the 7.4\,\unit{\mu m} channel. d: \acrshort{swd} formed by the 10.8\,\unit{\mu m} channel minus the 12.0\,\unit{\mu m} channel.
    }
    \label{fig:seviri_obs_example}
\end{figure}

An example of observations from \acrshort{seviri} is shown in fig.~\ref{fig:seviri_obs_example} for 15:00:00~\acrshort{utc} on 1\textsuperscript{st} June 2016. 
A visible composite (fig.~\ref{fig:seviri_obs_example}\,a) is constructed using the 1.64\,\unit{\mu m} and 0.81\,\unit{\mu m} near-infrared and 0.64\,\unit{\mu m} visible channels for the \acrshort{rgb} channels respectively. 
In this composite, ice clouds (which appear cyan) can be seen over central Africa and the southern Atlantic. 
Figure~\ref{fig:seviri_obs_example}\,b shows the 10.8\,\unit{\mu m} brightness temperature for the same scene, showing the coldest temperatures for the high ice clouds over central Africa. 
Two combinations of channels are used for the detection of anvil clouds. 
The \acrshort{wvd}, shown in fig.~\ref{fig:seviri_obs_example}\,c, consists of the 6.3\,\unit{\mu m} \acrshort{bt} minus the 7.4\,\unit{\mu m} \acrshort{bt}. 
The use of these channel combinations for the detection of thick and thin anvil clouds in satellite imagery is detailed in section~\ref{sec:abi_channels}.


\begin{figure}[tp]
    \centering
    \includegraphics[width=0.7\textwidth]{figures/chapter4_06.png}
    \caption[
    Comparison of the sensitivities of \acrshort{abi} (dashed lines) and \acrshort{seviri} (solid lines) to anvil clouds of different optical thickness
    ]{
    Comparison of the sensitivities of \acrshort{abi} (dashed lines) and \acrshort{seviri} (solid lines) to anvil clouds of different optical thickness, using the LibRadTran simulation of an anvil at 14\,\unit{km} as described in section~\ref{sec:theory_anvil}.
    }
    \label{fig:S_abi_seviri_anvil_sensitivity}
\end{figure}



\begin{figure}[tp]
    \centering
    \includegraphics[width=0.75\textwidth]{figures/chapter4_05.png}
    \caption[
    Comparison of the relative spectral response (RSR) functions for the \acrshort{goes}-16 \acrshort{abi} and Meteosat-11 \acrshort{seviri} thermal \acrshort{ir} channels
    ]{
    Comparison of the relative spectral response (RSR) functions for the \acrshort{goes}-16 \acrshort{abi} and Meteosat-11 \acrshort{seviri} thermal \acrshort{ir} channels. The \acrshort{lw} window channels on \acrshort{abi} (channels 13 and 15) have a wider spacing than those of \acrshort{seviri} (channels IR10.8 and IR12.0).}
    \label{fig:S_abi_seviri_rsr}
\end{figure}


Comparing \acrshort{seviri} to \acrshort{abi} (fig.~\ref{fig:S_abi_seviri_anvil_sensitivity}), the 10.8\,\unit{\mu m} \acrshort{bt} (top panel) and \acrshort{wvd} (middle panel) show very similar values for both instruments.
The 10.8\,\unit{\mu m} and 12.0\,\unit{\mu m} channels of \acrshort{seviri} have relatively wide wavebands (see fig.~\ref{fig:S_abi_seviri_rsr}), and as such are less sensitive to the presence of thin ice clouds (fig.~\ref{fig:S_abi_seviri_anvil_sensitivity}\,c), with a response about half that measured by \acrshort{abi}.
Combined with their high \acrshort{nedt} (see table~\ref{table:channel_nedt} and discussion in section~\ref{sec:abi_channels}), it is found that the detection of thin anvil is unreliable using \acrshort{seviri}, and so the thin anvil is not considered within this chapter.



Retrieved cloud properties, including optical thickness, effective radius, liquid/ice water path, \acrshort{ctt} and height, are provided by the \acrfull{cc4cl} algorithm \citep{sus_community_2018, mcgarragh_community_2018}. 
These properties are all retrieved at the same resolution as the input \acrshort{seviri} data. Broadband fluxes are derived using the BUGSRad radiative transfer model \citep{stephens_parameterization_2001} using input cloud properties from the \acrshort{cc4cl} retrieval and vertical temperature, moisture and trace gas profiles from ERA-5 \citep{hersbach_era5_2020}. 
The BUGSRad model provides \acrshort{toa} and \acrlong{boa} \acrshort{lw} and \acrshort{sw} radiative fluxes for both all-sky and clear-sky conditions. An example of these derived fluxes is shown in fig.~\ref{fig:seviri_flux_example}. 
Figure~\ref{fig:seviri_flux_example}\,a shows net \acrshort{toa} fluxes, with a net warming during the daytime on the Western side of the image, and a net cooling at night-time on the Eastern side. 
Figure~\ref{fig:seviri_flux_example}\,b shows the net \acrshort{toa} \acrshort{cre}, with a net cooling effect during the daytime and warming during the night-time for observed high clouds over central Africa. The \acrshort{sw} (fig.~\ref{fig:seviri_flux_example}\,c) and \acrshort{lw} (fig.~\ref{fig:seviri_flux_example}\,d) components of the \acrshort{cre} show that while the \acrshort{lw},
warming component has a smaller magnitude than the day-time, cooling \acrshort{sw} \acrshort{cre}, it remains constant during both day- and night-time.


\begin{figure}[tp]
    \includegraphics[width=\textwidth]{figures/chapter4_02.png}
    \caption[
    An example of the \acrshort{toa} \acrshort{cre} derived using the radiative flux model
    ]{
    An example of the \acrshort{toa} \acrshort{cre} derived using the radiative flux model, for the same time
    as shown in fig.~\ref{fig:seviri_obs_example} (15:00:00 \acrshort{utc} on 2016/6/01). a: net \acrshort{toa} radiative flux. b: net \acrshort{toa} \acrshort{cre}. c: \acrshort{sw} downwards \acrshort{cre}. d: \acrshort{lw} downwards \acrshort{cre}.
    }
    \label{fig:seviri_flux_example}
\end{figure}


\begin{figure}[tp]
    \includegraphics[width=\textwidth]{figures/chapter4_03.png}
    \caption[
    Validation of derived broadband fluxes against monthly \acrshort{ceres}-\acrshort{ebaf} \acrshort{cre}
    ]{
    Validation of derived broadband fluxes against monthly \acrshort{ceres}-\acrshort{ebaf} \acrshort{cre}. a.: The mean difference in net \acrshort{toa} \acrshort{cre} by 1\texttimes 1\textdegree grid square. b.: A comparison of observed \acrshort{toa} net \acrshort{cre} for \acrshort{seviri} against \acrshort{ceres}, with all locations in blue, and those where \acrshort{dcc} anvils are observed in red. c.: the mean difference in \acrshort{sw} \acrshort{toa} \acrshort{cre}. d.: comparison of \acrshort{sw} \acrshort{toa} \acrshort{cre} for \acrshort{seviri} and \acrshort{ceres}. e.: the mean difference in \acrshort{lw} \acrshort{cre}. f.: comparison of \acrshort{lw} \acrshort{toa} \acrshort{cre}. The stippling in a, c and e represents the locations in which \acrshort{dcc} anvils are observed, with the size of the dots corresponding to the number of observations. The solid lines in b, d and f show the linear regression for all locations (blue) and the locations of\acrshort{dcc} anvils (red) weighted by the number of observations.
    }
    \label{fig:flux_validation}
\end{figure}

Validation of the \acrshort{seviri} broadband fluxes was performed against monthly-mean observations of \acrshort{toa} broadband \acrshort{cre} from the \acrfull{ceres} \citep{loeb_clouds_2018} \acrfull{ebaf} climate data record. 
The results of this validation are shown in fig.~\ref{fig:flux_validation}. 
Monthly mean fluxes were calculated for \acrshort{seviri} by first calculating the mean daily fluxes over each 1\texttimes 1\textdegree grid square for days in which over 23 hours of observations are present, and then averaging these daily means over each month. 
Comparison of the net \acrshort{toa} \acrshort{cre} to \acrshort{ceres} revealed a bias of --\,1.87\,\unit{W m\textsuperscript{-2}} (fig.~\ref{fig:flux_validation}\,a,b), consisting of a \acrshort{sw} bias of --\,2.02\,\unit{W m^{-2}} (fig.~\ref{fig:flux_validation}\,c,d) and a \acrshort{lw} bias of $+$\,0.15\,\unit{W m\textsuperscript{-2}} (Fig~\ref{fig:flux_validation}\,e,f). 
Corrections for these biases have been applied uniformly to all further \acrshort{cre} values given in this chapter.


\section{Method}


\begin{figure*}[tp]
    \centering
    \includegraphics[width=0.75\textwidth]{figures/chapter4_04.png}
    \caption[
    Simulated sensitivity of the \acrshort{seviri} 10.8\,\unit{\mu m} \acrshort{bt} (top) and \acrshort{wvd} minus \acrshort{swd} (bottom) to anvil clouds of varying optical thickness at heights of 10, 12 and 14\,\unit{km}
    ]{
    Simulated sensitivity of the \acrshort{seviri} 10.8\,\unit{\mu m} \acrshort{bt} (top) and \acrshort{wvd} minus \acrshort{swd} (bottom) to anvil clouds of varying optical thickness at heights of 10, 12 and 14\,\unit{km} as described in section~\ref{sec:theory_anvil}. The grey dashed line shows the 241\,\unit{K} \acrshort{bt}. The grey region in the lower plot shows the range of temperatures in which the edge of the anvil is detected, as described in chapter~\ref{chp:tracking_method}.
    }
    \label{fig:S_anvil_sensitivity}
\end{figure*}

The detection and tracking of \acrshort{dcc}s are performed using the tobac-flow algorithm (see chapter~\ref{chp:tracking_method}, which has been designed specifically to track both isolated and clustered \acrshort{dcc}s in geostationary satellite imagery over their entire lifecycle.

Due to the lack of sensitivity of the \acrshort{seviri} \acrshort{swd} to thin ice clouds, only the thick portion of the anvil is tracked in this chapter.
The \acrshort{wvd} channel of \acrshort{seviri} is capable of detecting anvils with optical thicknesses of approximately 1.5$\pm$0.5 (see fig.~\ref{fig:S_anvil_sensitivity}).
The anvils tracked in this chapter have a median retrieved minimum optical depth of 1.45, although this value may be biased high as many anvils dissipate at night when accurate satellite retrievals of optical depth are not available.
While this sensitivity captures much of the \acrshort{cre} of \acrshort{dcc} anvils \citep{berry_cloud_2014} the long lifetimes of dissipating thin anvils may have a significant warming contribution to net anvil \acrshort{cre} \citep{horner_evolution_2023}.
As a result, it is expected that the anvil \acrshort{cre} measured in this study is biased low.

An example of the cores and anvils detected by the tobac-flow algorithm is shown in fig.~\ref{fig:seviri_detection}, at 3-hour intervals. 
In fig.~\ref{fig:seviri_detection}\,a, a large number of developing cores are seen over central Africa. 
In fig.~\ref{fig:seviri_detection}\,b, there are more developing cores over Western Africa as the pattern of initiation has shifted westward with the diurnal cycle.
In fig.~\ref{fig:seviri_detection}\,c,d fewer newly developing cores are observed later in the day, but the larger anvil clouds persist into the night-time.


\begin{figure}[tp]
    \includegraphics[width=\textwidth]{figures/chapter4_07.png}
    \caption[
    An example of the cores and anvils (detected by tobac-flow, shown at 3-hour time intervals
    ]{
    An example of the cores (red outline) and anvils (blue outline) detected by tobac-flow plotted over visible composite imagery from \acrshort{seviri}, shown at 3-hour time intervals. All times are given in \acrshort{utc}.
    }
    \label{fig:seviri_detection}
\end{figure}


Over the 4 months of the case study a total of 145,463 cores (of which 79,592 are associated with anvil clouds) and 35,941 anvils are tracked. 
Using the detected regions of both core and anvil components of tracked \acrshort{dcc}s, the cloud properties and \acrshort{cre} are calculated for each \acrshort{dcc} at each time step from the retrieval and broadband fluxes data. 
The resulting dataset allows us to analyse the properties of each \acrshort{dcc} over their lifetimes from a Lagrangian perspective.
While the studied domain contains both land and sea regions, only a small proportion of tracked \acrshort{dcc}s occurred over sea (11\%), and so analysis of land and oceanic \acrshort{dcc}s are not separated in this chapter.

\section{Results}

\subsection{Spatial and temporal distributions}

Figure~\ref{fig:seviri_map_dists}\,a shows the frequency of core detections for each 1\texttimes 1\textdegree\ grid square throughout the case study. 
The majority of observed convection occurs over the tropical rainforest regions. 
During the months of May-August, the \acrfull{itcz} is at its northernmost extent over Africa \citep{nicholson_itcz_2018}. 
The West African monsoon occurs during these months, with the primary band of convection located between 5-15\textdegree N \citep{nicholson_revised_2009}, which our observations agree with. 
The maximum frequency of convection is observed at around 6\textdegree N, 12\textdegree E over the Western High Plateau of Cameroon, with high frequencies of convection also observed over the Nigerian coastal plains to the West and the Jos Plateau in Northern Nigeria. 
High rates of convection are also observed over the coastal plains and inland highlands of Guinea, Sierra Leone and Liberia (5--12\,\textdegree N, 5--15\,\textdegree W).
Almost no convective activity is observed between 10\textdegree S and 20\textdegree S as, during the Northern hemisphere summer, this is the location of the descending branch of the Hadley cell which suppresses convection.


\begin{figure}[tp]
    \includegraphics[width=\textwidth]{figures/chapter4_08.png}
    \caption[
    Number of detected cores and average hour of core detection
    ]{
    a.: The total number of \acrshort{dcc} cores detected over the case study for each 1\texttimes 1\textdegree\ grid box. b.: The average hour of detection for the cores detected in each 1\texttimes 1\textdegree grid box. Grid boxes with a standard deviation greater than 6 hours are single-hatched, and greater than 12 hours cross-hatched.
    }
    \label{fig:seviri_map_dists}
\end{figure}


Figure~\ref{fig:seviri_map_dists}\,b shows the average time of detection for convection in each 1\texttimes 1\textdegree\ grid square. 
The average is calculated as the circular mean of the local solar times of core detection in the grid square. 
Grid squares with a standard deviation greater than 6 hours (indicating a broad spread of initiation times) are given single hatching, and those with standard deviations greater than 12 hours have cross-hatching. 
The most notable feature of the time of detection is the clear contrast between land and sea. 
Convection over the land tends to occur in the afternoon (15:00--18:00), whereas over the ocean it occurs between midnight and early morning (00:00--09:00). 
Furthermore, convection over land tends to occur in a fairly narrow range of times whereas over the ocean convection occurs throughout the diurnal cycle, resulting in the hatching applied to much of the ocean region. 
There is also a noticeable lake effect on the time of convection occurring over Lake Victoria (2\textdegree S, 34\textdegree E) and Lake Tanganyika (7\textdegree S, 31\textdegree E), with convection typically observed in the early morning.

When comparing the regions of Cameroon and Nigeria (4--10\textdegree N, 6--14\textdegree E), where the most growing cores were detected in fig.~\ref{fig:seviri_map_dists}\,a, with the average time of detection in fig.~\ref{fig:seviri_map_dists}\,b, it appears that the grid squares with more cores also tend to have an earlier average time of detection than the surrounding grid squares. 
Precipitation over the Nigerian plains and the Jos Plateau is linked to South-westerly winds bringing moist, warm air from the Gulf of Guinea \citep{vondou_seasonal_2010}. 
This warm air may then trigger convection both through the sea breeze effect and orographic lifting when it reaches the highlands, explaining both the higher frequency and earlier timing of convection compared to surrounding regions. 
A similar relationship between the high frequency of convection and earlier time of detection is also seen over the coastal region and adjacent highlands of Guinea, Sierra Leone and Liberia (5--12\textdegree N, 5--15\textdegree W) which may be due to the same mechanism.

It should be noted that due to the method of detection, cores that develop under existing anvils are less likely to be detected than those in clear sky regions. 
As a result, the occurrence of subsequent developing cores in organised systems may be underestimated, particularly in regions such as the Northern Sahel where a second, night-time peak of precipitation has been observed.

For all further analysis, only cores and anvils that are detected north of 15\textdegree S are considered so that the analysis is constrained to tropical \acrshort{dcc}s.

\subsection{Anvil Cloud Properties}

To investigate how the behaviour of \acrshort{dcc} anvils is affected by their organisation, observed anvils are grouped based on how many cores are associated with them, from isolated \acrshort{dcc}s with one core to highly-clustered \acrshort{dcc}s (such as tropical cloud clusters and \acrshort{mcs}s) with 10 or more cores. 
Anvils with 6--9 cores, and with 10 or more cores, are grouped together to ensure that these groups have a comparable number of members for analysis.

\begin{figure}[tp]
    \includegraphics[width=\textwidth]{figures/chapter4_09.png}
    \caption[
    Anvil statistics by number of associated cores for average maximum area, average lifetime, occurrence of anvils by number of cores, and percentage of total anvil coverage
    ]{
    Anvil statistics by number of associated cores for a.: average maximum area; b.: average lifetime; c.: the number of observed anvils by number of cores; and d.: percentage of total anvil coverage. Error bars in a and b show the standard error of the mean.
    }
    \label{fig:seviri_anvil_stats}
\end{figure}

Figure~\ref{fig:seviri_anvil_stats} shows properties related to the anvil area and lifetime linked to the number of cores. 
In fig.~\ref{fig:seviri_anvil_stats}\,a the average anvil maximum area is shown for each group. 
The maximum area increases approximately linearly with the number of cores, with increasingly clustered anvils having increasingly larger maximum areas, and highly clustered anvils having substantially larger anvils. 
Figure~\ref{fig:seviri_anvil_stats}\,b shows the average anvil lifetime compared to the number of cores. 
While the lifetime also increases with the number of cores, the difference between isolated and highly clustered anvils is proportionately smaller.



Figure~\ref{fig:seviri_anvil_stats}\,c shows the number of anvils observed with differing numbers of cores. 
The vast majority of all anvils observed are isolated \acrshort{dcc}s, with over 80\% having a single detected core. 
As the number of cores increases, the number of anvils detected decreases rapidly. 
However, when considering the large increase in both anvil area and lifetime with the number of cores, the total anvil coverage for highly clustered anvils is much larger (see fig.~\ref{fig:seviri_anvil_stats}\,d). 
Despite their high frequency, isolated \acrshort{dcc}s only account for 12\% of total anvil coverage, whereas highly clustered (10+ cores) account for over 50\%. 
Previous studies have found that despite being few in number, \acrshort{mcs}s account for the majority of precipitation in Western Africa \citep{vizy_understanding_2019}.

\begin{figure}[btp]
    \includegraphics[width=\textwidth]{figures/chapter4_10.png}
    \caption[
    Anvil statistics by number of cores for average anvil \acrshort{ctt} and average minimum anvil temperature
    ]{
    Anvil statistics by number of cores for a.: average anvil \acrshort{ctt}; and b.: average minimum anvil temperature. Error bars show the standard error of the mean.
    }
    \label{fig:seviri_anvil_ctt_stats}
\end{figure}


Figure~\ref{fig:seviri_anvil_ctt_stats}\,a shows the average mean \acrshort{ctt}, and fig.~\ref{fig:seviri_anvil_ctt_stats}\,b the average minimum \acrshort{ctt} for anvils with different numbers of cores. 
While the more clustered anvils have colder average anvil \acrshort{ctt}, this decrease plateaus below 220K indicating that the reduction in clear-sky cooling below this temperature may cap the anvil \acrshort{ctt} for larger \acrshort{dcc}s. 
The minimum observed \acrshort{ctt} within each anvil, however, are colder and show a greater difference with an increasing number of cores. 
The most clustered anvils tend to have a minimum \acrshort{ctt} of around 180\,\unit{K}, indicating the presence of overshooting tops and the most intense convection. 
These cold \acrshort{ctt} values have large uncertainty due to the high sensor noise-to-signal ratio at these cold temperatures.


\begin{figure}[tp]
    \includegraphics[width=\textwidth]{figures/chapter4_11.png}
    \caption[
    The distribution of time to coldest mean anvil \acrshort{ctt}, largest anvil area and time of anvil dissipation
    ]{
    The distribution of time to coldest mean anvil \acrshort{ctt} (orange), largest anvil area (green) and time of anvil dissipation (red) for anvils grouped by number of cores. The vertical lines show the mean time for each distribution.
    }
    \label{fig:seviri_lifetime_dists}
\end{figure}


\citet{futyan_deep_2007} divide the \acrshort{dcc} lifecycle into growing, mature and dissipating phases based on the time of observation of the coldest anvil \acrshort{ctt}, maximum anvil area and dissipation of the anvil. 
Figure~\ref{fig:seviri_lifetime_dists} shows the distribution of the time taken to reach each of these lifecycle milestones for anvils separated by the number of associated cores. 
For all cases, the average time of minimum anvil \acrshort{ctt} occurs before the maximum area, indicating that the anvils continue to grow beyond the maximum of convective activity. 
As the number of cores associated with each anvil increases, the time of the coldest \acrshort{ctt} and largest area occur proportionately earlier during the lifetime of the anvil. 
As a result, these more clustered anvils spend more of their lifetime existing with warming, shrinking anvils than the isolated \acrshort{dcc}s.


\begin{figure}[tp]
    \includegraphics[width=\textwidth]{figures/chapter4_12.png}
    \caption[
    The proportion of anvil lifetime spent in the growing, mature and dissipating phase
    ]{
    The proportion of anvil lifetime spent in the growing (orange), mature (green) and dissipating (red) phase, according to the criteria used by \citet{futyan_deep_2007}
    }
    \label{fig:seviri_lifetime_proportions}
\end{figure}



Figure~\ref{fig:seviri_lifetime_proportions}, compares the proportion of the overall anvil lifetime spent in each of the lifecycle phases defined by \citet{futyan_deep_2007} to the number of cores associated with the anvil. 
There is a clear trend that, as the number of cores increases, the proportion of the lifecycle spent in the growing phase decreases, and the proportion spent in the mature and dissipating phases increases.
Although this approach to classifying the lifecycle of anvil clouds is simplistic and does not capture the complexities of large, long-lived \acrshort{dcc}s which may go through multiple cycles of growth, dissipation and re-invigoration, it provides a useful simplification for comparing complex \acrshort{dcc} lifecycles \citep{roca_simple_2017}. 
The time of the coldest average \acrshort{ctt} will be when the \acrshort{lw} \acrshort{cre} of the anvil cloud is at its greatest, and so can help understand the evolution of the anvil \acrshort{cre} over its lifetime.


\subsection{Anvil \acrshort{cre}}

Using the broadband fluxes data in conjunction with the tracked \acrshort{dcc} dataset enables tracking of how the \acrshort{sw}, \acrshort{lw} and net \acrshort{cre} evolve over the lifetime of each tracked anvil.
Figure~\ref{fig:cre_lifecycle_examples} shows the time series of \acrshort{sw}, \acrshort{lw} and net \acrshort{cre} as well as the cumulative average \acrshort{cre} (the average of net \acrshort{cre} over anvil area and lifetime up until that point in time) for several different anvil lifecycles.
Note that all fluxes are \acrshort{toa} and measured in the downward direction, so a positive value represents warming and a negative value represents cooling.


\begin{figure}[tp]
    \centering
    \includegraphics[width=0.8\textwidth]{figures/chapter4_13.png}
    \caption[
    Anvil net, \acrshort{lw}, and \acrshort{sw} \acrshort{cre}, cumulative mean \acrshort{cre} over anvil lifetime
    ]{
    Anvil net, \acrshort{lw}, and \acrshort{sw} \acrshort{cre}, cumulative mean \acrshort{cre} over anvil lifetime and anvil area for a.: an isolated, short-lived (4-hour) \acrshort{dcc}; b.: a moderately clustered, 1-day long \acrshort{dcc}; and c.: a large, clustered, 4-day long \acrshort{dcc}. All times are the local solar time, to the nearest 5-minute interval. The black lines show the change in area of each \acrshort{dcc} over their lifecycle.
    }
    \label{fig:cre_lifecycle_examples}
\end{figure}


Figure~\ref{fig:cre_lifecycle_examples}\,a shows the case of an isolated, short-lived \acrshort{dcc}. 
The \acrshort{dcc} initiates during the daytime, during which the \acrshort{sw} \acrshort{cre} dominates and the net \acrshort{cre} is negative (cooling). 
However, towards the end of the four-hour lifecycle of the \acrshort{dcc}, it transitions to night-time and so while the \acrshort{sw} \acrshort{cre} reduces and eventually becomes zero, the \acrshort{lw} \acrshort{cre} dominates and the net \acrshort{cre} is positive (warming). 
While this period of warming moves the cumulative average \acrshort{cre} towards zero, it remains overall negative for the overall lifetime of the \acrshort{dcc} both due to the longer period spent during the daytime, and the larger area of the anvil cloud during this period.

Figure~\ref{fig:cre_lifecycle_examples}\,b shows the case of a longer-lived (22 hours), clustered \acrshort{dcc}. 
It initiates in the morning, and so the \acrshort{sw} cooling dominates for the first half of the anvil lifetime. 
Compared to the isolated \acrshort{dcc}, it exists for much longer during the night time, and so the cumulative average becomes positive over the full lifetime of the anvil cloud.

Figure~\ref{fig:cre_lifecycle_examples}\,c shows the case of a four-day, highly clustered convective event. 
In this case, the net \acrshort{cre} alternates between warming and cooling throughout the diurnal cycle. 
The cumulative \acrshort{cre} also alternates between overall warming and cooling throughout the lifetime of the anvil and results in a small net cooling effect.

In both the longer-lived cases (fig.~\ref{fig:cre_lifecycle_examples}\,b, c) the \acrshort{lw} \acrshort{cre} reduces towards the end of the anvil cloud lifetime. 
This may be reflective of the findings from fig.~\ref{fig:seviri_lifetime_dists} that the minimum average \acrshort{ctt} occurs before the mid-point of the cloud lifecycle for longer-lived systems. 
This reduction in \acrshort{lw} \acrshort{cre} may be due to a thinning of the anvil cloud (allowing increased \acrshort{lw} emission from the surface), or due to heating and stabilisation of the upper troposphere by the \acrshort{dcc}.
In addition, the cumulative radiative cooling of the anvil top may drive subsidence and reduce the cloud-top height of the anvil over time \citep{sokol_tropical_2020}.

Figure~\ref{fig:anvil_cre_dist} shows the distribution of net lifetime \acrshort{cre} for all tracked anvils. 
The overall negative average value of --\,0.94\,\textpm\,0.91\,\unit{W m^{-2}} is very close to zero considering the large spread in \acrshort{cre}. 
However, the distribution shows a bimodal structure, with two peaks at around +\,100\,\unit{W m^{-2}} (warming) and --\,180\,\unit{W m^{-2}} (cooling). 
The distribution is coloured according to the mean number of cores associated with the anvils in each bin of the distribution. 
Both the peaks of the distribution are mainly composed of isolated \acrshort{dcc}s which occur during the daytime (negative peak) or night-time (positive peak). 
The centre of the distribution---with average \acrshort{cre}s close to zero---shows a greater number of the clustered \acrshort{dcc}s with multiple cores which, due to their longer lifetime, tend to exist during both the day- and night time.


\begin{figure}[tp]
    \includegraphics[width=\textwidth]{figures/chapter4_14.png}
    \caption[
    The distribution of lifetime anvil \acrshort{cre} for all observed anvils
    ]{
    The distribution of lifetime anvil \acrshort{cre} for all observed anvils. The mean number of cores per anvil in each bin is indicated by the colour scale. The vertical dashed line shows the integrated mean \acrshort{cre} (over area and lifetime) over all anvils, weighted by the anvil areas (--0.94\,\textpm\,0.91\,\unit{W m^{-2}}).
    }
    \label{fig:anvil_cre_dist}
\end{figure}
\begin{figure}[btp]
    \includegraphics[width=\textwidth]{figures/chapter4_15.png}
    \caption[
    The distributions of mean anvil \acrshort{sw} \acrshort{cre} and \acrshort{lw} \acrshort{cre}
    ]{
    The distributions of mean anvil \acrshort{sw} \acrshort{cre} (a) and \acrshort{lw} \acrshort{cre} (b). The vertical dashed line shows the integrated mean \acrshort{cre} over all anvils (\acrshort{sw}: -135.0\,\textpm\,0.9\,\unit{W m^{-2}}, \acrshort{lw}: 134.0\,\textpm\,0.1\,\unit{W m^{-2}})
    }
    \label{fig:anvil_sw_lw_cre}
\end{figure}


In fig.~\ref{fig:anvil_sw_lw_cre} the \acrshort{cre} distribution is broken down into that of the \acrshort{sw} (fig.~\ref{fig:anvil_sw_lw_cre}\,a) and \acrshort{lw} (fig.~\ref{fig:anvil_sw_lw_cre}\,b) components. 
The \acrshort{sw} \acrshort{cre} shows a similar bimodal distribution to that of the net \acrshort{cre}, whereas the \acrshort{lw} distribution shows a normal distribution. 
The \acrshort{sw} \acrshort{cre} has a large peak at 0\,\unit{W m^{-2}} for \acrshort{dcc}s that occur during the night-time, and a broad peak centred around --\,300\,\unit{W m^{-2}} consisting of daytime \acrshort{dcc}s, with the average falling between the two. 
Note that the average for the \acrshort{lw} falls to the right of the peak of the distribution because the average is integrated over the anvil area and lifetime, and the largest and longest-lived anvils tend to have colder \acrshort{ctt} and hence larger \acrshort{lw} \acrshort{cre}.

Figure~\ref{fig:anvil_cre_time_vs_ctt} shows (a) the average instantaneous anvil \acrshort{cre} binned by the time of observation (local solar time) and mean anvil \acrshort{ctt}, and (b) the average lifetime anvil \acrshort{cre} binned by time of initial detection (local solar time) and mean anvil \acrshort{ctt}.
As expected, mean anvil \acrshort{cre} becomes more positive with decreasing \acrshort{ctt} due to increased \acrshort{lw} warming. 
However, the diurnal cycle of detection shows a much stronger contrast, with anvils detected during the daytime having a net cooling effect compared to those at night which have a net warming \acrshort{cre}. 
This diurnal cycle effect is stronger for those anvils with warmer average \acrshort{ctt}, generally representing isolated, shorter-lived \acrshort{dcc}s, and is weaker for colder anvil \acrshort{ctt}. 
Note also that in fig.~\ref{fig:anvil_cre_time_vs_ctt}\,b that the phase of the diurnal cycle shifts to earlier times of detection as average anvil \acrshort{ctt} become colder, as these \acrshort{dcc}s tend to have longer lifetimes.


\begin{figure}[tp]
    \includegraphics[width=\textwidth]{figures/chapter4_16.png}
    \caption[
    Average anvil \acrshort{cre} binned by the time of detection (local time) and mean anvil \acrshort{ctt}
    ]{
    (a) Average instantaneous anvil \acrshort{cre} binned by the time of observation (local solar time) and mean anvil \acrshort{ctt}. (b) Average lifetime anvil \acrshort{cre} binned by time of initial detection (local solar time) and mean anvil \acrshort{ctt}. Hashed regions in (b) show bins in which no anvils were detected.
    }
    \label{fig:anvil_cre_time_vs_ctt}
\end{figure}


It is apparent from figs.~\ref{fig:anvil_cre_dist} and \ref{fig:anvil_sw_lw_cre} that the observed neutral net anvil \acrshort{cre} is not only due to a balance between the \acrshort{sw} and \acrshort{lw}, but also from a balance of the cooling effect of daytime \acrshort{dcc}s and the warming effect of those occurring at night. 
If the number of \acrshort{dcc}s occurring during the daytime were to reduce then there could be a net warming effect without any change to the \acrshort{cre} of individual \acrshort{dcc}s.
As the diurnal cycle of convection over the ocean is nearly uniform, little impact on anvil \acrshort{cre} should be expected from changes in the time of convective initiation.
However, over land, where convective activity is much more common in the afternoon, changes in the diurnal cycle may have a much larger effect on anvil \acrshort{cre}.

Furthermore, fig.~\ref{fig:anvil_cre_time_vs_ctt}\,b highlights that differences in anvil temperature are linked to the diurnal cycle of anvil \acrshort{cre} as colder anvils tend to have longer lifetimes.
As a result, if warming surface temperatures lead to the invigoration of \acrshort{dcc}s, the warming effect would be larger than the \acrshort{lw} effect from the change in anvil temperature alone. 
Surface warming may also result in an earlier time of convective initiation, resulting in a cooling feedback.


\section{Summary}  %% \conclusions[modified heading if necessary]

By combining a novel cloud tracking algorithm with a new dataset of derived all-sky and clear-sky fluxes from geostationary satellite observations, \acrshort{dcc} anvils were detected and tracked along with their associated cores for both isolated and clustered \acrshort{dcc}s, and their properties, lifecycle and \acrshort{cre} investigated. 
As this study was performed using data from May-August (Northern hemisphere summer), the majority of convective activity was observed over the Guinea-Congo rainforest and Savanna regions, as the \acrshort{itcz} is at its northernmost extent.

The degree of convective clustering of each anvil is evaluated by measuring the number of cores it is associated with. 
As expected, anvils with the greatest number of cores---including \acrshort{mcs}s---have larger anvil areas, longer lifetimes and the coldest cloud tops. 
As a result, despite the majority of observed \acrshort{dcc}s being isolated, the highly clustered anvils make up most of the anvil coverage, and so cause most of the anvil impact over this region. 
In addition, the proportion of the lifecycle spent in the mature and dissipating phases increases with the number of cores, and the proportion spent in the growing phase decreases.

When investigating the net \acrshort{cre} of anvils, although the average \acrshort{cre} across all observed anvils is close to zero, few individual anvils have near zero \acrshort{cre} themselves. 
There exists a bimodal distribution of anvil \acrshort{cre}, with isolated \acrshort{dcc}s that exist during the daytime causing the negative (cooling) peak, and those that exist during the night-time causing the positive (warming) peak. 
The systems with near zero \acrshort{cre} tend to live longer with more cores, and exist during both the day- and night-time. 
As a result, when considering the magnitude of the anvil \acrshort{cre}, isolated \acrshort{dcc}s have an outsized contribution to the overall average anvil \acrshort{cre} of 21.4\% compared to their proportion of all anvil coverage (15.3\%) (see fig.~\ref{fig:S_contribution_to_net_cre}).


\begin{figure}[tp]
    \centering
    \includegraphics[width=0.75\textwidth]{figures/chapter4_17.png}
    \caption[
    The contribution to the net CRE for anvils with differing numbers of cores
    ]{
    The contribution to the net CRE for anvils with differing numbers of cores, which is defined as the sum of the absolute CRE multiplied by anvil area for all anvils with that number of cores, divided by the total for all anvils. Due to the large variance and magnitude of the CRE of isolated DCCs, they have a large impact on the net CRE balance despite their small area. 
    }
    \label{fig:S_contribution_to_net_cre}
\end{figure}


The net \acrshort{cre} measured over all anvils (--0.94\,\textpm\,0.91\,\unit{W m^{-2}}) is similarly near zero to many previous studies \citep[][e.g.]{ramanathan_cloud-radiative_1989, hartmann_effect_1992, hartmann_tropical_2016}.
It remains unclear however whether this is an equilibrium state or merely a coincidence.
The role of \acrshort{wv} in the tropical temperature profile, convective dynamics, the height of the convectively active tropopause and the tropical overturning circulation suggests that there may be a link between these factors.
Despite this, no mechanism restoring the tropical anvil \acrshort{cre} to near zero has been discovered.
Whether the anvil \acrshort{cre} should itself be considered near-zero is also debated.
\citet{stephens_cloudsat_2018} argues that, as tropical \acrshort{dcc}s often occur over low-level clouds which have a negative \acrshort{cre} which is cancelled out by the effect of \acrshort{dcc}s, meaning that the near zero \acrshort{toa} \acrshort{cre} is actually a result of a positive anvil \acrshort{cre}.
This line of reasoning has not become more widespread, however.

The interaction between the diurnal cycle of convection and \acrshort{dcc} lifetime plays a key role in the shape of the \acrshort{sw} anvil \acrshort{cre} distribution and is important to consider in regard to anvil \acrshort{cre} feedback. 
As the \acrshort{lw} \acrshort{cre} is normally distributed, a response to changing cloud top height or temperature may occur as a shift in the distribution. 
However, the bimodal distribution of the \acrshort{sw} \acrshort{cre} must result in more complex adjustments to shift the overall mean. 
As the position of the peak at 0 \,\unit{W m^{-2}} relating to night-time \acrshort{dcc}s is fixed, to change the overall average \acrshort{sw} \acrshort{cre} either the width of the distribution has to increase or decrease, or the number of \acrshort{dcc}s occurring during the day- or night-time has to increase. 
The former has important implications for the diurnal cycle of temperature in the tropics, and the latter for the diurnal cycle of convection, which, in turn, affects the anvil lifecycle.

Changes in the diurnal cycle of convection may not have a large impact on net anvil \acrshort{cre} over the ocean due to the mostly uniform occurrence of convection throughout the day.
Over land, however, the afternoon peak of convection at around 3\,pm solar time (see fig.~\ref{fig:seviri_map_dists}) coincides with a time at which anvil \acrshort{cre} is very sensitive to shifts in the diurnal cycle (fig.~\ref{fig:anvil_cre_time_vs_ctt}\,b).
Furthermore, a reduction or increase in the number of \acrshort{dcc}s occurring at a specific time of day may change the net \acrshort{cre} of anvils without any change in the \acrshort{cre} of individual \acrshort{dcc}s.

Diagnosing a diurnal-cycle-related anvil cloud feedback in climate models may however be difficult.
\citet{beydoun_dissecting_2021} found that changes in anvil lifetime contributed little to \acrshort{cre} feedbacks in a cloud-resolving radiative-convective-equilibrium model.
It is unclear how well the diurnal cycle of convection and convective lifecycle are represented in such a model, although convective-resolving models have been found to model these better than parameterised climate models \citep{prein_review_2015, feng_mesoscale_2023}.
Disentangling the impacts of convective processes and anvil cirrus processes on anvil lifecycle and \acrshort{cre} is also a key challenge.
Here, the use of model experiments such as \citet{gasparini_diurnal_2022} may help understand the impacts of each process on anvil \acrfull{cre} and the potential climate feedbacks.

% There are, however, a number of limitations in this study which present opportunities for future research. 
% Firstly, as this study only involved 4 months of data during the Northern Hemisphere summer, the impact of the seasonal cycle on the behaviour of \acrshort{dcc}s and their \acrshort{cre} could not be studied. 
% Furthermore, extending to a larger domain would allow investigation of regional differences, in particular the important land--sea contrast of deep convection \citep{takahashi_revisiting_2023}. 
% A major limitation of the \acrshort{seviri} data is its poor sensitivity to thin anvil cirrus, which has an important impact on net anvil \acrshort{cre} \citep{protopapadaki_upper_2017, horner_evolution_2023}.
% The flexible combined imager \citep{martin_fci_2021} aboard the third-generation Meteosat may allow better detection and study of thin anvil cirrus over tropical Africa in the near future.

% Cloud tracking provides a key capability for the study of deep convective anvil clouds \citep{gasparini_opinion_2023}.
% The ability to observe changes over the lifetime of an anvil cloud independently of changes in the microphysical or macrophysical properties of \acrshort{dcc}s.
% Further application of cloud tracking approaches may better our understanding of \acrshort{dcc} lifecycle, its relation to the diurnal cycle of radiation, and its response to a changing climate.
