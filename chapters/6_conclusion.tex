\chapter{Conclusion} \label{chp:conclusion}

\section{Summary}

\acrshort{dcc}s play a key role in the atmosphere, from extreme weather to the global overturning circulation and energy balance.
Understanding their response to global warming is vital for climate projection, but \acrshort{dcc} feedbacks remain uncertain \citep{sherwood_assessment_2020} and underestimated by current climate models \citep{hill_climate_2023}.
While traditional research on anvil feedbacks has focused on the large scale response of the climate (see section~\ref{sec:anvil_feedbacks}), a spate of novel research has highlighted the importance of convective processes the \acrshort{cre} of anvils \citep{raghuraman_observational_2024, sokol_greater_2024, mckim_weak_2024}.
The mechanisms through which convective processes influence anvil feedbacks remain unclear, in part due to a shortage of observational data connecting the properties of anvils to those of their convective cores \citep{gasparini_opinion_2023}.
With the growing interest in Lagrangian studies of \acrshort{dcc}s \citep{gasparini_what_2019, sokol_tropical_2020, bouniol_life_2021}, there is a need for improved observational constraints on the properties and behaviours of \acrshort{dcc}s.
In this thesis, novel cloud tracking techniques have been applied to geostationary satellite imagery to provide a new perspective on the relationship between convective cores and anvils.

Chapter~\ref{chp:tracking_method} focused on the development of a novel detection and tracking algorithm to better track the behaviour of \acrshort{dcc}s.
\acrshort{dcc} tracking algorithms have been developed for over 60 years, with applications for both weather and climate (see section~\ref{sec:tracking_timeline}).
Despite this, many algorithms still have drawbacks identified decades ago \citep{augustine_mesoscale_1988} that limit their application to developing research areas.
Several key requirements were defined to produce tracking data suitable for studying \acrshort{dcc}s across a wide range of scales and environments.
The algorithm would need to account for both isolated \acrshort{dcc}s and \acrshort{mcs}s, track both convective cores and anvils over the entirety of their lifetimes, and provide estimates of anvil area that are not dependent on other factors such as the anvil height.

Through theory developed using simple, 1D models (section~\ref{sec:detection_theory}), the use of various channel combinations available from modern geostationary imagers was developed to better identify \acrshort{dcc}s in observations.
To avoid the scaling issue between tracking of isolated systems and \acrshort{mcs}s, a semi-Lagrangian scheme was developed that removes the motion dependence from the detection and tracking task.
A detection method for growing convective cores based on cloud top cooling rates was developed, providing a good proxy for the vertical velocity of the core.
In addition, detection methods for both thick and thin anvils were developed using an edge detection scheme.
By avoiding the use of a fixed threshold, this method is more robust to bias than traditional methods, which also enables the detection of anvils both during the mature phase and while dissipating.
Evaluation against lightning observations showed that the combination of core and anvil detection provides a good balance of sensitivity and robustness,  with an F\textsubscript{1}-score of 0.91 providing a high degree of trust in the \acrshort{dcc}s detected by the algorithm.

In chapter~\ref{chp:lifecycle}, this tracking algorithm was applied over five years of geostationary satellite observations from \acrshort{goes}-16 \acrshort{abi}, addressing a key shortage of observational data linking core and anvil properties.
Analysis of this data shows a number of findings including the spatial distribution and land--sea contrasts of deep convection over North America.
Through Lagrangian analysis, the importance of the diurnal cycle both to the properties of \acrshort{dcc}s as well as to other studies could be investigated.
In particular, it was found that due to the contrasting diurnal cycles of convection over land and ocean, the average age of observed anvils varies widely throughout the day.
This contrast is particularly strong in the early afternoon, when \acrshort{dcc}s observed over land tend to be very young while those over the ocean are much older.
As this is the observing time for polar orbiting satellite missions such as those of the \acrshort{atrain}, it was found that snapshot observations from these sensors may find erroneous differences in the properties of \acrshort{dcc}s observed over land and sea due to these contrasting populations at certain times of day.
Combining the capabilities of polar-orbiting sensors with cloud tracking, as performed by \citet{elsaesser_simple_2022}, appears key to addressing these issues.

Previous research has found that the structure of anvil clouds is related to the convective intensity of their cores \citep{protopapadaki_upper_2017, takahashi_relationships_2017}.
In chapter~\ref{chp:anvil_structure}, the response of anvil cloud structure to both their observed intensity and organisation was evaluated using the dataset produced in the previous chapter.
The growing core cooling rate was used as a proxy for convective intensity, and the number of cores as a proxy for organisation, providing more direct proxies than those used in previous studies.
In addition, the thin anvil fraction was evaluated over the entire \acrshort{dcc} lifetime rather than at a single snapshot, capturing the growth and decay of the anvil.

While the relation between convective intensity and thin anvil fraction was found to be similar to that found in previous studies, increases in organisation had an opposing effect on the anvil structure.
The most intense anvils were found to have an increase in the thin anvil fraction of approximately 0.15 over the least intense anvils, whereas the most organised anvils had a fraction 0.12 less than the average for isolated \acrshort{dcc}s.
While convective organisation has a large impact on both the anvil area and lifetime, intensity was found to have little impact on the thick anvil lifetime and area but does increase that of the thin anvil, similar to findings by \citet{sokol_greater_2024}.
Intensity and organisation were also found to have opposing effects on the structure of the \acrshort{dcc} lifecycle.
While increasing intensity shortened the growing phase of the \acrshort{dcc} and lengthened the dissipating phase, organisation had the opposite effect.

To further investigate the cause of these differences, the change in thin anvil fraction over each \acrshort{dcc} lifetime was analysed.
In more intense anvils, differences were found both during the initial stage of anvil formation with more intense updrafts leading to a thinner anvil (indicating a direct impact of convective dynamics on the anvil structure) as well as during the dissipating phase, where the thin anvils produced by more intense \acrshort{dcc}s took longer to dissipate.
For more organised anvils, the main differences were observed during the mature phase of the \acrshort{dcc}, during which time organised systems showed a thickening of the anvil, possibly due to the impact of mesoscale circulations produced by organised convective systems.
While these processes can be observed directly, the use of lifecycle analysis combined with an understanding of how different processes affect \acrshort{dcc}s at different stages of their lifetimes provides insight into what effects these processes may have.
This lifecycle analysis may also be useful in constraining the behaviour of \acrshort{dcc}s in convective resolving models against observations.

 
The novel approach used in this chapter has allowed new results to be found that were not identified in the previous studies.
Explicitly tracking the full lifetime of the anvil cloud results in attributing a larger proportion of thin anvil area than that seen by snapshot observations, and reveals changes caused by an increase in the anvil lifetime.
Furthermore, the explicit tracking of convective cores associated with anvils showed the response of thin anvil area to organisation, whereas without this the trends of organised convection would be hidden due to the majority of observed \acrshort{dcc}s being isolated systems.
In addition, we show that the intensity and organisation of convection have opposing effects on both the lifecycle and structure of \acrshort{dcc}s. 
The thin cirrus produced by \acrshort{dcc}s may have a warming effect on the climate, and so understanding how the area and lifetime of these clouds change with the properties of convection is important to understanding \acrshort{dcc} feedbacks. 
As both the intensity of \acrshort{dcc}s and the frequency of organised convective systems are expected to increase with global warming, so further study of the response of thin cirrus to convective activity is vital.

% \subsection{Chapter~\ref{chp:radiative_effect}: the influence of the diurnal cycle on the \acrshort{cre} of tropical anvil clouds}

Traditionally, considerations of anvil feedbacks have only considered their bulk properties with a present day net anvil \acrshort{cre} of near zero.
In chapter~\ref{chp:radiative_effect}, the cloud tracking method developed in chapter~\ref{chp:tracking_method} is combined with retrieved cloud properties and radiative fluxes from \acrshort{seviri} observations over Africa to investigate how the \acrshort{cre} of individual \acrshort{dcc}s contributes to the net observed \acrshort{cre}. 
Together, this allows us to explicitly measure the \acrshort{cre} of \acrshort{dcc}s over their lifecycle, and analyse how they change with \acrshort{dcc} properties.
The distributions of observed \acrshort{dcc}s show both similarities and differences from those seen in chapter~\ref{chp:lifecycle}.
A similar land--sea contrast was shown in the timing of convective initiation.
This contrast was also seen in the time of initiation over the African Great Lakes (particularly Lake Victoria), which may indicate similar mechanisms in action.
The small number of \acrshort{dcc}s due to the limited extent of the case study prevented further investigation into land--sea contrasts.
The observed \acrshort{dcc} lifecycle, shown in fig.~\ref{fig:seviri_lifetime_proportions}, shows notable differences to those found in chapter~\ref{chp:anvil_structure}, which may have several reasons.
Firstly, there may be a difference in the lifecycle of \acrshort{dcc}s between the tropics and extra-tropics.
However, this difference may have been due to the lower sensitivity of \acrshort{seviri} for observing thin anvils.
This lower sensitivity reduces the length of time over which the anvil is observed, increasing the proportion of the lifecycle observed in the growing phase.
Furthermore, this effect is expected to be greater for isolated \acrshort{dcc}s as the thick anvil dissipates faster than for larger \acrshort{mcs}s.
Finally, the use of retrieved \acrshort{ctt}, rather than observed \acrshort{bt}, may cause a difference.
An increase in stratospheric \acrshort{wv}, along with the thinning of the anvil cloud, will cause the observed \acrshort{bt} of the anvil to increase over time.
However, the retrieved \acrshort{ctt} will remove these effects.
As a result, the method of \citet{futyan_deep_2007} for measuring anvil lifecycle may produce a longer growing phase if \acrshort{ctt} is used instead of \acrshort{bt}.

The main aim of this chapter was the investigation of how the diurnal cycle affects the \acrshort{cre} of individual \acrshort{dcc}s.
Historically, studies into the \acrshort{cre} of tropical anvils have averaged over all areas of anvil cloud without tracking individual \acrshort{dcc}s \citep{ramanathan_cloud-radiative_1989, hartmann_effect_1992, stephens_cloudsat_2018}.
While the overall average \acrshort{cre} of our observed \acrshort{dcc}s confirmed the neaar neutral \acrshort{cre} of tropical anvil clouds with a net value of --0.94\,\textpm\,0.91\,\unit{W m^{-2}}, through the use of tracking we could study how the properties of individual \acrshort{dcc}s contribute to this mean state.

The \acrshort{cre} of \acrshort{dcc}s changes over their lifetime both due to the changing properties of the anvil cloud and the timing within the diurnal cycle.
While both the \acrshort{lw} and \acrshort{sw} \acrshort{cre} reduce over the lifetime of the \acrshort{dcc}, the \acrshort{sw} \acrshort{cre} is also largely dependent on the time within the diurnal cycle.
These two effects lead to the bimodal distribution for the \acrshort{cre} of individual \acrshort{dcc}s, as shown in fig.~\ref{fig:anvil_cre_dist}.
We can identify three groups of \acrshort{dcc}s that contribute to this distribution.
The largest group is that of isolated \acrshort{dcc}s, which form the two peaks of the distribution.
Typically, these either initiate in the early afternoon and have a negative \acrshort{cre} due to the large \acrshort{sw} effect during the daytime, or initiate in the early evening and instead have a positive \acrshort{cre} as they exist at night.
The second group consists of moderately clustered \acrshort{dcc}s which tend to exist for longer than isolated \acrshort{dcc}s but less than a day.
As these systems tend to initiate in the evening, they mostly exist at night and therefore predominantly have a \acrshort{lw} warming effect.
Finally, \acrshort{mcs}s which exist for multiple days tend to have a neutral \acrshort{cre} as their \acrshort{sw} and \acrshort{lw} effects average out across the diurnal cycle.

The behaviour of these different \acrshort{dcc}s may influence how they respond to climate change.
Organised \acrshort{dcc}s, including \acrshort{mcs}s, may primarily respond through changes in their \acrshort{lw} \acrshort{cre} including \acrshort{fat} and anvil area as \acrshort{sw} changes tend to average out over their lifecycle.
On the other hand, isolated \acrshort{dcc}s tend to have \acrshort{ctt} warmer than the top of the convectively active layer (220\,\unit{K}), and so \acrshort{fat} may not apply to them in the same manner.
The large variance in the \acrshort{cre} of individual isolated \acrshort{dcc}s means that they have a substantially larger influence on the net anvil \acrshort{cre} balance than their area suggests.
As a result, the response of these systems to global warming may have a larger effect on the overall anvil feedback than has been previously considered, and, in particular, they may be sensitive to changes in the diurnal cycle.


\section{Discussion \& Future work}

In recent years there has been an increased focus on the study of anvil lifecycle \citep{wall_life_2018, sokol_tropical_2020} and radiative effects \citep{bouniol_life_2021}.
The response of \acrshort{dcc}s to climate change is both important and uncertain \citep{sherwood_assessment_2020, hill_climate_2023}.
Improving our understanding of cirrus cloud properties and lifecycles is vital to improving this situation \citep{sullivan_ice_2021, gasparini_opinion_2023}.

The key results of this thesis are in the new observational insights into the lifecycle, structure and \acrshort{cre} of \acrshort{dcc}s and their anvils.
The development of a novel tracking algorithm allows both isolated and organised \acrshort{dcc}s to be tracked across their entire lifetime across a wide range of scales.
Furthermore, by tracking both cores and anvil clouds, we can link the behaviours of the convective cores to the evolution of the anvil clouds, and provide observational evidence for how these properties affect each other.

Chapters~\ref{chp:lifecycle}, and \ref{chp:radiative_effect} showed significant differences in the lifecycles of \acrshort{dcc}s over the land and the ocean.
In particular, these differences may influence \acrshort{dcc} cores and anvils to different extents.
Several processes, occurring over a range of scales, may influence the properties of observed \acrshort{dcc}s.
Lake effects, such as those seen over the African Great Lakes in chapter~\ref{chp:radiative_effect}, may provide a useful case study for understanding these differences, as while their local and surface properties may be similar to that of oceans, the larger environmental conditions may be more affected by the adjacent land regions.

In chapter~\ref{chp:anvil_structure} we found contrasting impacts of convective intensity and organisation on the structure and lifecycle of anvil clouds. 
Analysing how the structure of these \acrshort{dcc}s evolved over their lifetime provided key insights into which convective and anvil processes lead to the observed changes in thick thin anvils.
These contrasts may help further understand the uncertainties in anvil radiative feedbacks, and, in particular, help explain how changes in convection may affect anvil optical properties \citep{mckim_weak_2024}.
In addition, it highlights the importance of isolated \acrshort{dcc}s to the total anvil \acrshort{cre}, motivating further study into how their behaviour differs from organised convection.
Further investigating the contrast between the effects of intensification and organisation on the structure of \acrshort{dcc}s anvils may provide insight into the net changes of anvil structure with climate change. However, while with these structural changes alone we would expect more intense \acrshort{dcc}s to be warming, and more organised \acrshort{dcc}s to be cooling, the changes in \acrshort{bt} and the diurnal cycle observed for both of these processes may dominate any changes in \acrshort{cre}.

A key area of opportunity is in combining the investigation of the impacts of intensity and organisation on anvil structure in chapter~\ref{chp:anvil_structure} with the investigation of anvil \acrshort{cre} in chapter~\ref{chp:radiative_effect}.
While the potential impact of structural changes on anvil \acrshort{cre} was discussed in chapter~\ref{chp:anvil_structure}, explicitly resolving the impact on \acrshort{cre} and comparing it to that due to changes in anvil height and diurnal cycle would advance our understanding of these processes collectively.
To do so, the issues with tracking thin anvils in \acrshort{seviri} observations would need to be overcome.
A larger cloud tracking study, combining these elements, investigating how the microphysical, macrophysical properties and structure of \acrshort{dcc}s change over their lifecycle and throughout the diurnal cycle, and then demonstrating the effect of these changes on the \acrshort{cre} of the anvil across both thick and thin parts and along its entire lifetime, could greatly reduce uncertainties in anvil cloud feedbacks.

\acrshort{dcc}s play a key role in many atmospheric processes, and their response to global warming is both uncertain \citep{sherwood_assessment_2020} and underestimated by current climate models \citep{hill_climate_2023}.
While the new generation of convective resolving models may help to address this problem \citep{stevens_added_2020}, these models still do not fully resolve the issues in simulating the dynamics of deep convection \citep{jeevanjee_vertical_2017} and the behaviour of anvil cirrus \citep{sullivan_ice_2021}.
Observational studies of \acrshort{dcc}s using cloud tracking approaches have been highlighted as a key method to addressing some of these issues \citep{gasparini_opinion_2023}.
In this thesis, we have developed and utilised a novel cloud tracking algorithm to better investigate the properties of \acrshort{dcc} cores and anvils as seen by the latest generation of geostationary weather satellites.
Key results from chapters~\ref{chp:lifecycle}, \ref{chp:anvil_structure} and \ref{chp:radiative_effect} have shown new findings on the properties, structure and \acrshort{cre} of \acrshort{dcc} anvils across their lifecycle, which provide insight into how \acrshort{dcc}s may respond to changes in convective processes and climate change.
The ability to track the properties of \acrshort{dcc}s over their entire lifetimes has been key to these findings.
Understanding the lifecycle of \acrshort{dcc}s from a Lagrangian perspective has been highlighted as a key need for future satellite missions \citep{vandenheever_tropical_2023}, convective resolving model development \citep{prein_kmscale_2024}, and upcoming observation campaigns, and so cloud tracking has a key role to play in future studies of deep convection.
