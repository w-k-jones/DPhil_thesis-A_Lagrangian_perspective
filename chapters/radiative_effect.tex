\chapter{A Lagrangian Perspective on the Lifecycle and Cloud Radiative Effect of Deep Convective Clouds Over Africa}

% \textbf{Abstract}

% The anvil clouds of tropical deep convection have large radiative
% effects in both the shortwave and longwave spectra with magnitudes of
% both over 100Wm\textsuperscript{-2}. Despite this, due to the opposite
% sign of these fluxes the net, long-term average of anvil cloud radiative
% effect (CRE) over the tropics is neutral. Much of the related literature
% has focused on the interaction between anvil cloud temperature and its
% impact on the longwave flux component, and how this may respond to
% climate change. However, tropical deep convection over land has a strong
% diurnal cycle which may couple with the shortwave component of anvil
% cloud radiative effect.

% To study the interaction between deep convective cloud (DCC) lifecycle
% and CRE, we investigate the behaviour of both isolated and clustered
% DCCs in a 4-month case study over sub-Saharan Africa (May-August 2016).
% Using a novel cloud tracking algorithm, we detect and track growing
% convective cores and their associated anvil clouds using geostationary
% satellite observations from Meteosat SEVIRI. Retrieved cloud properties
% and derived broadband radiative fluxes are provided by the CC4CL
% algorithm. By combining the cloud properties with the tracked DCCs, we
% produce a dataset of anvil cloud properties along their lifetimes. While
% the majority of DCCs tracked in this dataset are isolated, with only a
% single core, the overall coverage of anvil clouds is dominated by those
% of clustered, multi-core anvils due to their larger areas and lifetimes.

% We find that the distribution of anvil cloud CRE of our tracked DCCs has
% a bimodal distribution. The interaction between the lifecycles of DCCs
% and the diurnal cycle results in a wide range of CRE in the SW
% component, while the LW component remains in a comparatively narrow
% range of values. Very few individual anvil clouds have net CRE close to
% zero, and those which do tend to be larger, longer-lived, and have more
% cores. Despite this, we find that the total integrated anvil cloud CRE
% across all tracked DCCs is indeed zero within our range of uncertainty
% (0.86±0.91 Wm-2). Changes in the lifecycle of DCCs, such as shifts in
% the time of triggering, or the length of the dissipating phase, could
% have large impacts on the SW anvil CRE and lead to complex responses
% that are not considered by theories of LW anvil CRE feedbacks.

\section{Introduction}

Deep convective clouds (DCCs) play a key role in the tropical
atmosphere. Forming the ascending branch of the Hadley cells near the
equator, DCCs are critical to the circulation and heat transfer of the
tropics (Riehl and Malkus 1958, Weisman 2015). DCCs are also a cause of
extreme weather events including floods, lightning and hail (Westra et
al., 2014). Mesoscale convective systems (MCSs) -- large, long-lived
convective systems in which the anvils of multiple convective cores
combine into a single, large ``cloud shield'' (Chen and Houze 1996,
Houze 2004, Roca et al. 2017) -- are responsible for the majority of
precipitation in the tropics (Tan et al. 2015). Changes in the behaviour
of DCCs with climate change have the potential for major impacts on the
atmosphere, weather and society.

DCCs also exert a key influence on the temperature of the tropics
through their cloud radiative effect (CRE). Due to their size, height
and depth, DCC anvils have large radiative effects in both the shortwave
(SW) and longwave (LW), with both having average magnitudes in excess of
100Wm\textsuperscript{-2} (Hartmann 2016, Wall and Hartmann 2018).
However, due to the opposite signs of these two components, the average
anvil CRE in the tropics is approximately zero (Ramanathan et al. 1989,
Hartmann et al. 1992, Stephens et al. 2018). Radiation is also key to
the lifecycle of DCCs. Over land, convection is typically initiated by
the heating of the surface and lower troposphere by solar SW radiation,
resulting in a peak of convective activity in the late afternoon. Over
the ocean however, convection is often triggered by LW cooling of the
upper troposphere, and so convective activity occurs more frequently in
the morning. However, the occurrence of convection is more uniform
throughout the diurnal cycle compared to that over land (Taylor et al.
2017). Radiation also has an impact on DCC lifecycle through the
differential heating of the anvil cloud, which destabilises the anvil
cloud leading to dissipation due to entrainment and evaporation.
However, SW heating of the anvil cloud top during daytime acts to
stabilise and delay this process, leading to differences in anvil
lifetime depending on the diurnal cycle (Sokol and Hartmann, 2020).

There are a number of hypotheses regarding the CRE of tropical anvil
clouds that consider whether the neutral CRE of tropical anvils is the
result of a feedback mechanism. The thermostat hypothesis proposes that
in a warmer environment anvil clouds produce thicker cirrus which acts
to cool the tropics through increased SW reflectance (Ramanathan and
Collins, 1991). The Iris hypothesis proposes an alternate negative
feedback mechanism in which anvil clouds reduce in area in response to
warming surface temperature (due to an increase in precipitation
efficiency), resulting in greater LW emission from the surrounding clear
regions (Lindzen et al. 2001), however support for this effect is
disputed (Del Genio and Kovari 2002, Lin et al. 2004). On the other
hand, the fixed anvil temperature (FAT) hypothesis argues that the anvil
cloud top temperature remains constant in a warming climate due to the
tendency of tropical anvil clouds to detrain at the level at which water
vapour cooling becomes inefficient (around 200 K). This would result in
a positive feedback mechanism whereby anvil clouds become higher to
maintain the same cloud top temperature and so have a larger LW CRE due
to the warmer surface temperatures (Hartmann and Larson, 2002). While
there is evidence that this is the case for the largest DCC anvils, it
is debated whether these observations are due to FAT or due to fixed
tropopause temperature (FiTT) (Seeley et al. 2019). The proportionately
higher anvil temperature (PHAT) hypothesis argues instead that while
anvil clouds will become higher with warming temperatures, an associated
increase in static stability results in warmer anvil temperatures and a
reduced LW response compared to FAT (Zelinka and Hartmann, 2010). This
reduced response more closely matches the LW response of tropical clouds
in global climate models. Some observations of tropical anvil clouds
have instead suggested that warming of the surface invigorates
convection, leading to higher and colder anvil CTT and a stronger LW
warming response (Igel et al., 2014).

These hypotheses however only consider changes in anvil properties such
as area, reflectance and temperature, and FAT and PHAT only consider the
LW feedback. Changes to the lifecycle and diurnal cycle of deep
convection may also be an important factor, in particular when
considering the SW feedback. Deep convection over land may be perturbed
in particular by factors which influence SW fluxes, such as aerosols.
Observations have shown that the increase in tropical precipitation can
be attributed to an increase in the frequency of deep convection, rather
than an intensification of individual DCCs (Tan et al. 2015).

In this article, we investigate how the lifecycle of deep convection
impacts the cloud radiative effect of anvil clouds. To do so, we use a
novel cloud tracking methodology in conjunction with derived all-sky and
clear-sky radiative fluxes to characterise the CRE over the lifecycles
of individual anvil clouds. This methodology is applied to 4 months of
data produced for the ESA Cloud-CCI+ project over sub-Saharan Africa.
This dataset allows us to investigate both the CRE of individual CREs,
as well as the net anvil CRE over the entire region. We find that the
overall distribution of anvil CRE is determined by the relationship
between DCC lifecycle and the diurnal cycle of the SW CRE, and discuss
the implications of this for the response of DCCs to a changing climate.

\section{Data}

For this case study, we used data from the Spinning Enhanced Visible and
Infra-Red Imager (SEVIRI) (Schmid 2000) aboard the Meteosat Second
Generation (MSG) Meteosat-11 satellite, which is in a geostationary
orbit above the equator at 0°W. We use data from 4 months (May-August
2016) over sub-Saharan Africa (approximately 18°W-46°E, 31°S-15°N) at
the full resolution of SEVIRI, as well as retrieved cloud properties and
derived broadband fluxes produced by the ESA Cloud-CCI+ project.
Calibrated brightness temperature (BT) from SEVIRI is used by the
tracking algorith, and calibrated reflectances and BT are used by the
cloud retrieval.

SEVIRI is a visible and infra-red (IR) radiometer with a nadir spatial
resolution of 3km and a temporal sampling time of 15 minutes for the
full earth disc. SEVIRI has 12 channels across the visible, \acrshort{nir} and
thermal-IR spectrum, with one being a high-resolution visible channel
with a nadir resolution of 1km. A brief overview of these channels,
along with which are used for tracking DCCs and the cloud properties
retrieval, is provided in table 1:

\begin{table}[t]
\centering
\begin{tabular}{lllcc}
\tophline
Channel & Wavelength (\unit{\mu m}) & Description & Tracking & Retrieval\tabularnewline
\middlehline
1 & 0.64 & Visible & & \checkmark\tabularnewline
2 & 0.81 & \acrshort{nir} & & \checkmark\tabularnewline
3 & 1.64 & \acrshort{nir} & & \checkmark\tabularnewline
4 & 3.92 & \acrshort{nir} Window & & \checkmark\tabularnewline
5 & 6.25 & Upper troposphere \acrshort{wv} & \checkmark & \checkmark\tabularnewline
6 & 7.35 & Lower troposphere \acrshort{wv} & \checkmark & \checkmark\tabularnewline
7 & 8.70 & Mid-IR window & &\tabularnewline
8 & 9.66 & Ozone & &\tabularnewline
9 & 10.8 & Clean LW window & \checkmark & \checkmark\tabularnewline
10 & 12.0 & Dirty LW window & \checkmark & \checkmark\tabularnewline
11 & 13.4 & CO\textsubscript{2} & & \checkmark\tabularnewline
12 & 0.6-0.9 & High-resolution visible & &\tabularnewline
\bottomhline
\end{tabular}
\caption[
SEVIRI channels and their use in the DCC tracking algorithm and cloud properties retrieval
]{
SEVIRI channels and their use in the DCC tracking algorithm and cloud properties retrieval.
}
\label{table:seviri_channels}
\end{table}



An example of observations from SEVIRI is shown in fig. 1 for 15:00:00
UTC on 1\textsuperscript{st} June 2016. A visible composite (Fig. 1a) is
constructed using the 1.64\,\unit{\mu m} and 0.81\,\unit{\mu m} near-infrared and 0.64\,\unit{\mu m} visible
channels for the red, green and blue channels respectively. In this
composite, ice clouds -- which appear cyan -- can be seen over central
Africa and the southern Atlantic. Fig. 1b shows the 10.8\,\unit{\mu m} brightness
temperature for the same scene, showing the coldest temperatures for the
high ice clouds over central Africa. Two combinations of channels are
used for the detection of anvil clouds. The \acrshort{wvd}
(WVD), shown in Fig. 1c, consists of the 6.3\,\unit{\mu m} brightness temperature
minus the 7.4\,\unit{\mu m} brightness temperature. In clear skies the WVD is
negative, with values around -20 to -15 K, due to the higher, and thus
colder, emission height of the 6.3\,\unit{\mu m} channel. In high, thick clouds,
however, the temperatures of the 6.3 and 7.4\,\unit{\mu m} channels converge and so
the WVD becomes closer to 0. In the cases of the highest clouds, the WVD
can become positive due to emission from stratospheric WV in the 6.3\,\unit{\mu m}
channel. The split window difference (SWD), shown in Fig. 1d, consists
of the 10.8\,\unit{\mu m} BT channel minus the 12.0\,\unit{\mu m} channel. While the SWD is
sensitive to near-surface WV due to absorption in the 12.0\,\unit{\mu m} channel, it
is also sensitive to thin ice clouds due to the difference in emissivity
of ice particles between the two channels. While for thick clouds the
SWD will be 0 K, for thin ice clouds the lower emission height of the
10.8\,\unit{\mu m} BT channel results in a positive value of 5 K. SEVIRI has wider
wavebands for these two channels compared to newer sensors such as the
GOES-16 ABI, and as such is less sensitive to the presence of thin ice
clouds.

Retrieved cloud properties -- including optical thickness, effective
radius, liquid/ice water path, cloud top temperature and height -- are
provided by the community cloud retrieval for climate (CC4CL) algorithm
(Sus et al. 2018, McGarragh et al. 2018). These properties are all
retrieved at the same resolution as the input SEVIRI data. Broadband
fluxes are derived using the BUGSRad radiative transfer model (Stephens
et al. 2001) using input cloud properties from the CC4CL retrieval and
vertical temperature, moisture and trace gas profiles from ERA-5
(Hersbach et al. 2020). The BUGSRad model provides top-of-atmosphere
(ToA) and bottom-of-atmosphere (BoA) LW and SW radiative fluxes for both
all-sky and clear-sky conditions. An example of these derived fluxes is
shown in figure 2. Figure 2a shows net ToA fluxes, with a net warming
during the daytime on the Western side of the image, and a net cooling
at night-time on the Eastern side. Figure 2b shows the net ToA CRE, with
a net cooling effect during the daytime and warming during the
night-time for observed high clouds over central Africa. The SW (Fig.
2c) and LW (Fig. 2d) components of the CRE show that while the LW,
warming component has a smaller magnitude than the day-time, cooling SW
CRE, it remains constant during both day- and night-time.

Validation of the SEVIRI broadband fluxes was performed against
calibrated monthly-mean observations of ToA broadband CRE from the
clouds and the earth's radiant energy system (CERES) (Loeb et al., 2018)
Energy Balanced and Filled (EBAF) climate data record. The results of
this validation are shown in Fig. 3. Monthly mean fluxes were calculated
for SEVIRI by first calculating the mean daily fluxes over each 1x1°
grid square for days in which we have over 23 hours of observations, and
then averaging these daily means over each month. Comparison of the net
ToA CRE to CERES revealed a bias of -3.67Wm\textsuperscript{-2} (Fig.
3a,b), consisting of a SW bias of -3.04Wm\textsuperscript{-2} (Fig.
3c,d) and a LW bias of -0.63Wm\textsuperscript{-2} (Fig. 3e,f). These
biases have been accounted for in all further CRE values given in this
article.

\section{Method}

Detection and tracking of DCCs was performed using the tobac-flow
algorithm (Jones et al. 2023), which has been designed specifically to
track both isolated and clustered DCCs in geostationary satellite
imagery over their entire lifecycle. While geostationary satellite
imagery provides high-resolution observations over large domains and
long time periods, which is ideal for studying deep convection, the
inability of passive remote sensing to observe convective updrafts
directly makes the detection and tracking of DCCs difficult.

Algorithms for the detection and tracking of DCCs can generally be split
into two groups. Firstly, those designed for tracking deep convective
cores, or isolated DCCs, such as Cb-TRAM (Zinner et al. 2008, 2011) or
tobac (Heikenfeld et al. 2019, Sokolowsky et al. 2023). These algorithms
work by detecting regions of convective updraft or a proxy (such as
cloud top cooling rate), and then treating these regions as point-like
objects that are advected over time. Secondly, those designed for
tracking mesoscale convective systems such as PyFLEXTRKR (Feng et al.
2022), TAMS (Núñez Ocasio et al. 2020) or TOOCAN (Fiolleau and Roca,
2013). These algorithms detect large regions of cold cloud tops
indicating anvils, and then link them over time by overlapping regions
at subsequent time steps. There is no ``best'' method for tracking all
types of convection however (Lakshmanan and Smith, 2009). The algorithms
for tracking isolated convective cells perform worse for clustered
convection when the motion and shape of the DCC cannot be adequately
represented as a single vector. On the other hand, the MCS tracking
algorithms perform worse for smaller, isolated DCCs as the motion of the
anvil between time steps may mean it does not overlap with the previous
step.

To approach the challenge of tracking both isolated DCCs and large,
clustered systems, we address the role of cloud motion in the scaling
problem. tobac-flow first estimates the motion of DCCs at each pixel
using an optical-flow algorithm. Then, using these estimated motion
vectors, we construct a semi-Lagrangian framework in which to perform
the detection and tracking. This framework removes the problem of DCC
motion, allowing us to track both isolated and large DCCs at the same
time.

We detect growing convective cores where we observe regions of rapid
cooling in the 10.8\,\unit{\mu m} BT channel and the \acrshort{wvd} (WVD);
the difference between the 6.2\,\unit{\mu m} and 7.3\,\unit{\mu m} channels. Using both
differences allows us to detect growing DCCs close to the surface and
continue tracking them into the upper troposphere. We classify a core as
a region of cooling temperature that has existed for at least 15 minutes
and has cooled by at least 8K in a 15-minute period. This threshold
provides a strong indicator of intense convective activity (Roberts and
Rutledge 2003), and so provides an accurate detection of growing DCCs.
Starting from these convective cores, we then detect the surrounding
anvil cloud using the WVD field (Müller et al. 2018, 2019) and continue
to detect the anvil until its dissipation, even after the core is no
longer visible. Each anvil cloud can be associated with multiple cores,
allowing us to identify cases of clustered convection. As we detect the
cores based on cloud-top cooling, however, we can only detect the cores
themselves during the growing phase, and cannot detect cores which occur
underneath cold, high, anvil clouds. Due to the lack of sensitivity of
the SEVIRI SWD to thin ice clouds, we only detect and track the thick
portion of the anvil in this article.

An example of the cores and anvils detected by the tobac-flow algorithm
is shown in Fig. 4, at 3-hourly intervals. In Fig. 3a, we see a large
number of developing cores over central Africa. In Fig. 3b, we see more
developing cores over Western Africa as the pattern of initiation has
shifted with the diurnal cycle. In Fig. 3c,d we observe fewer new
developing cores later in the day, but the larger anvil clouds persist
into the night-time.

Over the 4-month period of the case study, we detect a total of 145,463
cores (of which 79,592 are associated with anvil clouds) and 35,941
anvils after quality controls have been applied. Using the detected
regions of core and anvil clouds, the cloud properties and CRE are
calculated for each DCC at each time step from the retrieval and
broadband fluxes data. The resulting dataset allows us to analyse the
properties of each DCC over their lifetimes from a Lagrangian
perspective.

\section{Results}

\subsection{Spatial and temporal distributions}

Figure 5(a) shows the frequency of core detections for each 1x1° grid
square over the period of the case study. The majority of observed
convection occurs over the Guinea-Congo region. During the months of
May-August, the ITCZ is at its northernmost extent over Africa
(Nicholson 2018). The West African monsoon occurs during these
months, with the primary band of convection located between 5-15°N
(Nicholson 2009), which our observations agree with. We observed the
maximum frequency of convection at around 6°N, 12°E over the Western
High Plateau of Cameroon, with high frequencies of convection also
observed over the Nigerian coastal plains to the West and the Jos
Plateau in Northern Nigeria. High rates of convection are also observed
over the coastal plains and inland highlands of Guinea, Sierra Leone and
Liberia (5-12°N, 5-15°W)

Figure 5(b) shows the average time of detection for convection in each
1x1° grid square. The average is calculated as the circular mean of the
local solar times of core detection in the grid square. Grid squares
with a standard deviation greater than 6 hours (indicating a broad
spread of initiation times) are given single hatching, and those with
standard deviations greater than 12 hours have cross-hatching. The most
notable feature of the time of detection is the clear contrast between
land and sea. Convection over the land tends to occur in the afternoon
(15:00-18:00), whereas over the ocean it occurs between midnight and
early morning (00:00-09:00). Furthermore, convection over land tends to
occur in a fairly narrow range of times whereas over the ocean
convection occurs throughout the diurnal cycle, resulting in the
hatching applied to much of the ocean region. There is also a noticeable
lake effect on the time of convection occurring over Lake Victoria (2°S,
34°E) and Lake Tanganyika (7°S, 31°E), with convection typically
observed in the early morning.

When we compare the regions of Cameroon and Nigeria (4-10°N, 6-14°E),
where we detect the most cores in Fig. 5(a), with the average time of
detection in Fig. 5(b), we see that the grid squares with more cores
also tend to have an earlier average time of detection than the
surrounding grid squares. Precipitation over the Nigerian plains and the
Jos plateau is linked to South-westerly winds bringing moist, warm air
from the Gulf of Guinea (Vondou et al. 2010). This warm air may then
trigger convection both through the sea breeze effect and orographic
lifting when it reaches the highlands, explaining both the higher
frequency and earlier timing of convection compared to surrounding
regions. A similar relationship between the high frequency of convection
and earlier time of detection is also seen over the Guinea/Sierra
Leone/Liberia region (5-12°N, 5-15°W), and may be due to the same
mechanism.

It should be noted that due to the method of detection, cores that
develop under existing anvils are less likely to be detected than those
in clear sky regions. As a result, we may underestimate the occurrence
of later occurring cores, particularly in regions such as the Northern
Sahel where a second, night-time peak of precipitation has been
observed.

For all further analysis, we consider only cores and anvils that are
detected north of 15°S in order to constrain our analysis to tropical
DCCs.

\subsection{Anvil Cloud Properties}

To investigate how the behaviour of DCC anvils is affected by their
organisation, we group observed anvils based on how many cores are
associated with them, from isolated DCCs with one core to
highly-clustered DCCs (such as tropical cloud clusters and MCSs) with 10
or more cores. Anvils with 6-9 cores, and with 10 or more cores, are
grouped together to ensure that these groups have a comparable number of
members for analysis.

Figure 6 shows properties related to the anvil area and lifetime linked
to the number of cores. In Fig. 6(a) we show the average anvil maximum
area for each group. We find that the maximum area increases
approximately linearly with the number of cores, with increasingly
clustered anvils having increasingly larger maximum areas, and highly
clustered anvils having substantially larger anvils. Figure 6(b) shows
the average anvil lifetime compared to the number of cores. While the
lifetime also increases with the number of cores, the difference between
isolated and highly clustered anvils is proportionately smaller.

Figure 6(c) shows the number of anvils observed with differing numbers
of cores. We see that the vast majority of all anvils observed are
isolated DCCs, with over 80\% having a single detected core. As the
number of cores increases, the number of anvils detected decreases
rapidly. However, when considering the large increase in both anvil area
and lifetime with the number of cores, the total anvil coverage for
highly clustered anvils is much larger (see Fig. 6(d)). Despite their
high frequency, isolated DCCs only account for 12\% of total anvil
coverage, whereas highly clustered (10+ cores) account for over 50\%.
Previous studies have found that despite being few in number, MCSs
account for the majority of precipitation in Western Africa (Vizy and
Cook 2018).

Figure 7(a) shows the average mean cloud top temperature (CTT), and Fig.
7(b) the average minimum CTT for anvils with different numbers of cores.
While the more clustered anvils have colder average anvil CTT, this
decrease plateaus around 225K. However, the absolute minimum observed
CTT within each anvil shows a much greater difference with an increasing
number of cores. The most clustered anvils tend to have a minimum CTT of
around 185K, indicating the presence of overshooting tops and the most
intense convection. However, care should be taken when interpreting such
retrieved CTT due to the large uncertainty associated with sensor noise
at these cold temperatures.

Futyan and Del Genio (2007) divide the DCC lifecycle into growing,
mature and dissipating phases based on the time of observation of the
coldest anvil CTT, maximum anvil area and dissipation of the anvil. In
Fig. 8 we show the distribution of the time taken to reach each of these
lifecycle milestones for anvils separated by the number of associated
cores. For all cases, the average time of minimum anvil CTT occurs
before the maximum area, indicating that the anvils continue to grow
beyond the maximum of convective activity. As the number of cores
associated with each anvil increases, the time of the coldest CTT and
largest area occur proportionately earlier during the lifetime of the
anvil. As a result, these more clustered anvils spend more of their
lifetime existing with warming, shrinking anvils than the isolated DCCs.

In Fig. 9, we compare the proportion of the overall anvil lifetime spent
in each of the lifecycle phases defined by Futyan and Del Genio to the
number of cores associated with the anvil. There is a clear trend that,
as the number of cores increases, the proportion of the lifecycle spent
in the growing phase decreases, and the proportion spent in the mature
and dissipating phases increases. Although this approach to classifying
the lifecycle of anvil clouds is simplistic and does not capture the
complexities of large, long-lived DCCs which may go through multiple
cycles of growth, dissipation and re-invigoration, it can provide a
useful perspective when considering the LW CRE of DCCs. The time of the
coldest average CTT will be when the LW CRE of the anvil cloud is at its
greatest, and so can help understand the evolution of the anvil CRE over
its lifetime.

\subsection{Anvil Cloud CRE}

Using the broadband fluxes data in conjunction with the tracked DCC
dataset, we are able to track how the SW, LW and net CRE evolve over the
lifetime of each tracked anvil. Figure 6 shows the time series of SW, LW
and net CRE as well as the cumulative average CRE for a number of
different cases of anvil lifecycles. Note that all fluxes are
top-of-atmosphere and measured in the downward direction, so a positive
value is warming and a negative value represents cooling.

Figure 10(a) shows the case of an isolated, short-lived DCC. The DCC
initiates during the daytime, during which the SW CRE dominates and the
net CRE is negative (cooling). However, towards the end of the four-hour
lifecycle of the DCC, it transitions to night-time and so while the SW
CRE reduces and eventually becomes zero, the LW CRE dominates and the
net CRE is positive (warming). While this period of warming moves the
cumulative average CRE towards zero, it remains overall negative for the
overall lifetime of the DCC both due to the longer period spent during
the daytime, and the larger area of the anvil cloud during this period.

Figure 10(b) shows the case of a longer-lived (22 hours), clustered DCC.
It initiates in the morning, and so the SW cooling dominates for the
first half of the anvil lifetime. Compared to the isolated DCC, it
exists for much longer during the night time, and so the cumulative
average becomes positive over the full lifetime of the anvil cloud.

Figure 10(c) shows the case of a four-day, highly clustered convective
event. In this case, we see the net CRE alternative between warming and
cooling throughout the diurnal cycle. The cumulative CRE also alternates
between overall warming and cooling throughout the lifetime of the anvil
and is results in a small net cooling effect.

We see in both the longer-lived cases (fig. 10(b) and 10(c)) that the LW
CRE reduces towards the end of the anvil cloud lifetime. This may be
reflective of the findings from Fig. 8 that the minimum average CTT
occurs before the mid-point of the cloud lifecycle for longer-lived
systems. In addition, the accumulated LW cooling of the anvil top may
drive subsidence and reduce the cloud-top height of the anvil over time
(Sokol and Hartmann, 2020)

Figure 11 shows the distribution of net lifetime CRE for all tracked
anvils. The overall negative average value of -8.17±0.85
Wm\textsuperscript{-2} is approximately zero when considering the
negative bias in the broadband flux dataset. However, the distribution
shows a bimodal structure, with two peaks at around
+100Wm\textsuperscript{-2} (warming) and -180Wm\textsuperscript{-2}
(cooling). The distribution is coloured according to the mean number of
cores associated with the anvils in each bin of the distribution. Both
the peaks of the distribution are mainly composed of isolated DCCs which
occur during the daytime (negative peak) or night-time (positive peak).
The centre of the distribution -- with average CREs close to zero --
shows a greater number of the clustered DCCs with multiple cores which,
due to their longer lifetime, tend to exist during both the day- and
night time.

In Fig. 12 we break down the CRE distribution into that of the SW (fig.
12(a)) and LW (fig. 12(b)) components. The SW CRE shows a similar
bimodal distribution to that of the net CRE, whereas the LW distribution
shows a normal distribution. The SW CRE has a large peak at
0Wm\textsuperscript{-2} for DCCs that occur during the night-time, and a
broad peak centred around -300Wm\textsuperscript{-2} consisting of
daytime DCCs, with the average falling between the two. Note that the
average for the LW falls to the right of the peak of the distribution
because the average is integrated over the anvil area and lifetime, and
the largest and longest-lived anvils tend to have colder CTT and hence
larger LW CRE.

Figure 13 shows the average net anvil CRE binned by intervals of time of
detection and mean anvil CTT. We see that, as expected, mean anvil CRE
becomes more positive with increasing CTT. However, the diurnal cycle of
detection shows a much stronger contrast, with anvils detected during
the daytime having a cooling effect compared to those at night. This
diurnal cycle effect is stronger for those anvils with cooler average
CTT, generally representing isolated, shorter-lived DCCs, and is weaker
for colder anvil CTT. Note also that the phase of the diurnal cycle
shifts to earlier times of detection as average anvil CTT become colder,
as these DCCs tend to have longer lifetimes.

For anvil cloud CRE to be radiatively balanced, sufficient DCCs must
initiate during the daytime, cooling region shown in Fig. 13 to balance
the warming effect of DCCs initiating during the rest of the diurnal
cycle. As anvil temperatures become colder, this region becomes narrower
and shifts earlier in the day, due to both the increased LW CRE of
colder anvil clouds and also due to the tendency of these anvils to have
longer lifetimes. As a result, if warming surface temperatures lead to
the invigoration of DCCs, the warming effect we would see would be
larger than just that due to the change in anvil temperature alone. To
restore the net anvil CRE to zero, the distribution of DCCs may need to
shift earlier in the diurnal cycle, leading to large changes in the
patterns of convection and precipitation. This may also further affect
the anvil lifetime, due to the differences in the anvil subsidence
between day- and nighttime (Sokol and Hartmann, 2020)

\subsection{Summary}

By combining a novel cloud tracking algorithm with a new dataset of
derived all-sky and clear-sky fluxes from geostationary satellite
observations, we were able to detect and track DCC anvils and their
associated cores for both isolated and clustered DCCs and investigate
their properties, lifecycle and CRE. As this study was performed using
data from May-August (Northern hemisphere summer), we observed the
majority of convective activity over the Guinea-Congo rainforest and
Savanna regions, as the ITCZ is at its northernmost extent.

We evaluate the degree of convective clustering of each anvil by
measuring the number of cores it is associated with. We find that, as
expected, anvils with the greatest number of cores -- including MCSs --
have larger anvil areas, longer lifetimes and the coldest cloud tops. As
a result, despite the majority of observed DCCs being isolated, the
highly clustered anvils make up most of the anvil coverage, and so cause
most of the anvil impact over this region. We also find that the
proportion of the lifecycle spent in the mature and dissipating phases
increases with the number of cores, and the proportion spent in the
growing phase decreases.

When looking into the net CRE of anvils, we find that, although the
average CRE across all observed anvils is approximately zero, few anvils
have near zero CRE themselves. We find a bimodal distribution of anvil
CRE, with isolated DCCs that occur during the daytime causing the
negative (cooling) peak, and those which occur during the night-time
causing the positive (warming) peak. The systems with near zero CRE tend
to live longer with more cores, and exist during both the day- and
night-time. As a result, when considering the magnitude of the anvil
CRE, isolated DCCs have an outsize contribution to the overall average
anvil CRE of 18.7\% compared to their proportion of all anvil coverage
(11.9\%).

The interaction between the diurnal cycle of convection and DCC lifetime
plays a key role in the shape of the SW anvil CRE distribution and is
important to consider in regard to anvil CRE feedback. As the LW CRE is
normally distributed, a response to changing cloud top height or
temperature may occur as a shift in the distribution. However, the
bimodal distribution of the SW CRE must result in more complex
adjustments to shift the overall mean. As the position of the peak at 0
Wm\textsuperscript{-2} relating to night-time DCCs is fixed, to change
the overall average SW CRE either the width of the distribution has to
increase or decrease, or the number of DCCs occurring during the day- or
night-time has to increase. The former has important implications for
the diurnal cycle of temperature in the tropics, and the latter for the
diurnal cycle of convection.

This highlights topics of interest for future research. Firstly, as this
study only involved 4 months of data during the Northern Hemisphere
summer, we were not able to investigate the impact of the seasonal cycle
on the behaviour of DCCs and their CRE. Extending this research to a
full year of data over a larger domain would allow investigation of
seasonal and regional differences, in particular also over the oceans.
Secondly, an investigation of atmospheric heating effects from DCCs. SW
and LW fluxes have notably different effects on atmospheric heating, and
so although the top-of-atmosphere flux may be in balance for tropical
DCCs, changes in the LW or SW CRE may have resulted in different heating
profiles and diurnal cycles. In particular, anvil heating by LW and SW
is important to anvil lifetime and dependent on the diurnal cycle
(Harrop and Hartmann 2018, Sokol and Hartmann 2020), and so gaining a
better understanding of this may aid in understanding the lifecycle of
both isolated and clustered convection in the tropics. Finally,
investigating the response of DCC CRE to perturbations of the diurnal
cycle of deep convection. Anthropogenic effects such as biomass burning
aerosol radiative effect and land use change may change the diurnal
cycle of deep convection over land, and as shown this may have a large
influence on the ToA CRE. Understanding how the CRE of deep convection
responds to changes in the diurnal cycle may be key to understanding
future changes in deep convection in the tropics.

\includegraphics[width=6.26389in,height=4.08637in]{media/image1.png}

Figure 1: Example observations from the Meteosat SEVIRI instrument at
15:00:00 UTC on 2016/6/01. a: A visible composite formed using the 1.6,
0.81 and 0.64 micron channels as the red, green and blue channels
respectively, with 10.8 micron brightness temperature during the
night-time. The scene shows a cluster of cold cloud tops (cyan) over
central Africa and over the Southern Atlantic. b: 10.8 micron brightness
temperature. c: \acrshort{wvd} (WVD) formed by the 6.3 micron
channel minus the 7.4 micron channel. d: Split window difference (SWD)
formed by the 10.8 micron channel minus the twelve micron channel.

\includegraphics[width=6.25465in,height=3.94375in]{media/image2.png}

Figure 2: An example of the top of atmosphere (ToA) cloud radiative
effect (CRE) derived using the radiative flux model, for the same time
as shown in figure 1 (15:00:00 UTC on 2016/6/01). a: net ToA radiative
flux. b: net ToA CRE. c: shortwave (SW) upwards CRE. d: longwave (LW)
upwards CRE.

\includegraphics[width=6.26389in,height=6.37778in]{media/image3.png}

Figure 3: Validation of derived broadband fluxes against monthly
CERES-EBAF CRE. a.: The mean difference in net ToA CRE by 1x1° grid
square. b.: A comparison of observed ToA net CRE for SEVIRI against
CERES, with all locations in blue, and those where we observe DCC anvils
in red. c.: the mean difference in SW ToA CRE. d.: comparison of SW ToA
CRE for SEVIRI and CERES. e.: the mean difference in LW CRE. f.:
comparison of LW ToA CRE. The stippling in a,c,e represents the
locations in which we observe DCC anvils, with the size of the dots
corresponding to the number of observations. The solid lines in b,d,f
show the linear regression for all locations (blue) and the locations in
we observe dcc anvils (red), weighted by the number of observations.

\includegraphics[width=6.26389in,height=5.42099in]{media/image4.png}

Figure 4: Detected cores (red outline) and anvils (blue outline) at
3-hour time intervals. All times are given in UTC.

\includegraphics[width=6.26in,height=8.15139in]{media/image5.png}

Figure 5: number of detected cores (a) and average hour of core
detection (b) by 1x1° grid box. Grid boxes in (b) with a standard
deviation greater than 6 hours are single-hatched, and greater than 12
hours cross-hatched

\includegraphics[width=6.26389in,height=6.10531in]{media/image6.png}

Figure 6: Anvil statistics by number of associated cores for average
maximum area (a), average lifetime (b), occurrence of anvils by number
of cores (c), and percentage of total anvil coverage (d). Error bars (a,
b) show standard error of the mean

\includegraphics[width=5.80457in,height=3.31042in]{media/image7.png}

Figure 7: Anvil statistics by number of cores for average anvil cloud
top temperature (a), and average minimum anvil temperature (b). Error
bars (a,b) show standard error of the mean

\includegraphics[width=6.26389in,height=4.85862in]{media/image8.png}

Figure 8: The distribution of time to coldest mean anvil CTT (orange),
largest anvil area (green) and time of anvil dissipation (red) for
anvils grouped by number of cores. The vertical lines show the mean time
for each distribution.

\includegraphics[width=6.26389in,height=5.03567in]{media/image9.png}

Figure 9: The proportion of anvil lifetime spent in the growing
(orange), mature (green) and dissipating (red) phase, according the
criteria used by Futyan and Del Genio (2007)

\includegraphics[width=5.78218in,height=9.05037in]{media/image10.png}

Figure 10: Anvil net, LW, and SW CRE, accumulated mean CRE over anvil
lifetime and anvil area for (a) an isolated, short-lived (4-hour) DCC,
(b)a moderately clustered, 1-day long DCC, and (c) a large, clustered,
4-day long DCC. All times are the local solar time, to the nearest 5
minute interval

\includegraphics[width=6.26389in,height=4.64149in]{media/image11.png}

Figure 11: The distribution of lifetime anvil CRE for all observed
anvils. The mean number of cores per anvil in each bin is indicated by
the colour scale. The vertical dashed line shows the integrated mean CRE
over all anvils, weighted by the anvil areas (0.86±0.91
Wm\textsuperscript{-2}).

\includegraphics[width=6.26389in,height=3.46779in]{media/image12.png}

Figure 12: The distributions of mean anvil SW CRE (a) and LW CRE (b).
The vertical dashed line shows the integrated mean CRE over all anvils
(SW: -133.9±0.9 Wm\textsuperscript{-2}, LW: 134.8±0.1
Wm\textsuperscript{-2})

\includegraphics[width=6.26389in,height=4.98401in]{media/image13.png}

Figure 13: Average anvil CRE binned by the time of detection (local
time) and mean anvil CTT.

\end{document}
