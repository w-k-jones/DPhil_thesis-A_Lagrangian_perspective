\begin{abstract}

% \acrfullpl{dcc}---characterised by their great height, intense updraft velocities and large areas---are responsible for a wide range of extreme weather and subsequent natural disasters including extreme rainfall, flooding, hail, lightning and tornadoes. Furthermore, deep convection plays a key role in many parts of the climate, including the atmospheric general circulation, the radiative balance of the top-of-atmosphere, the energetic balance of the troposphere and the hydrological cycle. Global warming is expected to increase the intensity and frequency of deep convection.  Understanding the behaviour of these systems is therefore vital to forecasting both weather in the present day and the future climate.

% Studying the behaviour of \acrshort{dcc}s is made difficult by both their extreme dynamics, and also the range of scales over which their effects occur. \acrshort{dcc}s involve typical vertical velocities on the order of 10\,\unit{ms\textsuperscript{-1}}, substantially larger than seen elsewhere in the atmosphere. Their properties scan a range of scales from the order of 100\,\unit{m} for thermals within updrafts, 10\,\unit{km} for convective cores, 100s--1000s\,\unit{km} for their associated anvil clouds and even further for their coupling with the wider circulation of the atmosphere. Climate models have traditionally struggled to represent many of the behaviours of \acrshort{dcc}s, and while km-scale convective resolving models have improved in this regard, there are still large remaining uncertainties.

% Geostationary satellite observations offer a unique capability to observe the entire extent and lifecycle of \acrshort{dcc}s. Modern geostationary satellite instruments can observe processes occurring at single km and minute scales across domains spanning many thousands of km over multiple years. Algorithms to detect and track \acrshort{dcc}s in geostationary satellite imagery have been used to capture the dynamical behaviour of \acrshort{dcc}s and represent them in a Lagrangian framework since the 1970s. Despite several decades of development, however, it remains difficult to track \acrshort{dcc}s across the range of scales at which they occur on account of their motion. There has been parallel development of algorithms designed to detect individual convective cores, and those designed to track large, \acrfullpl{mcs}. In chapter~\ref{chp:tracking_method}, we develop a novel method to detect and track both the cores and anvils of both isolated and mesoscale convective storms. This approach utilised optical flow to estimate the cloud motion, removing the scale dependence of convective tracking. We demonstrate that, by combining the detection of both cores and anvil clouds, we can more accurately track \acrshort{dcc}s across their entire lifetimes.

% North America experiences a wide array of convective storms in both tropical and extra-tropical environments, over the ocean and over land. In chapter~\ref{chp:lifecycle}, we apply the approach developed in the previous chapter to five years of imagery from the advanced baseline imager to study the behaviour of \acrshort{dcc} cores and anvils in this region. Using this dataset, we investigate the spatial, diurnal and seasonal distributions of both cores and anvils, and show the relation between them and their contrasting behaviour between land and ocean. In addition, we show that the intensity and organisation of convection have opposing effects on both the lifecycle and structure of \acrshort{dcc}s. The thin cirrus produced by \acrshort{dcc}s may have a warming effect on the climate, and so understanding how the area and lifetime of these clouds change with the properties of convection is important to understanding \acrshort{dcc} feedbacks. Both the intensity of \acrshort{dcc}s and the frequency of organised convective systems are expected to increase with global warming, so further study of the response of thin cirrus to convective activity is vital.

% Despite their large reflectance of \acrfull{sw} radiation, and absorption of \acrfull{lw} radiation, tropical anvil clouds have historically been found to have a net \acrfull{cre} close to zero. Previous studies have not, however, investigated how the \acrshort{cre} of individual \acrshort{dcc}s combine to result in this average. Furthermore, the main focus of anvil cloud feedbacks has been on the \acrshort{lw} component of their \acrshort{cre}. The \acrshort{sw} aspect may, due to the strong diurnal cycle of convective behaviour, also have a significant impact, particularly if the diurnal cycle of deep convection changes in response to global warming. In chapter~\ref{chp:radiative_effect}, we combine the tracking of \acrshort{dcc}s with the retrieval of cloud radiative fluxes from geostationary observations over Africa to investigate the \acrshort{cre} of \acrshort{dcc}s changes due to their lifecycle. We find that the large \acrshort{mcs}s which contribute the majority of anvil cloud cover tend to have \acrshort{cre} close the zero. However, smaller, isolated \acrshort{dcc}s tend to have large negative or positive \acrshort{cre} if they occur at night- or daytime respectively, which results in a bimodal distribution of \acrshort{sw} anvil \acrshort{cre}. Despite the wide range of the anvil \acrshort{cre} distribution, we find that the overall average is indeed zero. While changes in the diurnal cycle would have little effect on the \acrshort{cre} of \acrshort{mcs}s, they could have large impacts on the \acrshort{cre} of isolated \acrshort{dcc}s. Furthermore, the large non-zero \acrshort{cre} of these smaller systems means that the overall average anvil \acrshort{cre} is more sensitive to their changes than previously considered.

% Through the use of a novel storm tracking algorithm, we have used geostationary satellite imagery to study to properties of deep convective cores across a wide range of scales. In doing so we have found two key results that require further investigation for understanding future changes in deep convection. Further investigating the contrast between the effects of intensification and organisation on the structure of \acrshort{dcc}s anvils may provide insight into the net changes of anvil structure with climate change. Furthermore, the bimodal \acrshort{cre} distribution of tropical anvil clouds highlights that the contribution of isolated \acrshort{dcc}s is more important to the anvil \acrshort{cre} balance than previously thought, and that these systems may be more sensitive to climate change. Combining research on both these effects may help understand whether the net impact of the response of \acrshort{dcc}s to global warming will be a warming or cooling feedback. Future satellite missions to investigate the processes of deep convection are planning to include cloud tracking to provide insight into the lifecycle of \acrshort{dcc}s, and it is clear that continued development of this technology is required to further the study of these important atmospheric phenomena.


\end{abstract}
