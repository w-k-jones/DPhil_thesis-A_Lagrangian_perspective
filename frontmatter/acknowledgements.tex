\begin{acknowledgements}

I would like to begin by thanking my supervisor, Philip Stier, for his excellent mentorship and allowing me the freedom to explore the many aspects of cloud and storm physics.
The path to writing this thesis has been far from the easiest one, and I would not have made it without such a supportive supervisor.

I would also like to thank Matt Christensen, my co-advisor, for his support and for first getting me interested in the study of clouds through the lens of satellite observations.

The \textit{tobac} development team, including, but not limited to, Max Heikenfeld, Fabian Senf, Sean Freeman, Julia Kukulies and Kelcy Brunner have greatly helped shape the development of the approaches used within the study the properties of deep convective clouds.

I would like to thank Sue van den Heever and Graeme Stephens for many insightful discussions on the behaviour of deep convection, which in particular inspired the contents of chapter~\ref{chp:anvil_structure}.

I thank Martin Stengel and the ESA Cloud-CCI+ team for providing the cloud retrievals that made chapter~\ref{chp:radiative_effect} possible.

I am very grateful to Don Grainger and Johannes Quaas for volunteering their time as my examiners.

I give thanks to my parents, Rosie and Stephen, for their continued guidance, love and support, even if they couldn't convince me that there were options other than Physics.

Finally, but most importantly, I would like to thank my partner Cat for her love and support during times both happy and difficult.
Let's hope that the stormy skies are behind us now.

% I would like to acknowledge the financial support provided by the European Research Council, Horizon 2020 RECAP (grant no. 724602) and the Natural Environment Research Council Environmental Research Doctoral Training Programme to fund these studies. Furthermore, I would like to acknowledge the financial support received from the European Space Agency through the Cloud\_cci project (contract no.: 4000128637/20/INB) which funded the research carried out in chapter~\ref{chp:radiative_effect}.

\end{acknowledgements}